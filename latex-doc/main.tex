\documentclass[10pt, openany, a5paper]{book} % Define um documento do tipo "livro", com fonte 10pt, formato A5 e permitindo capítulos começarem em qualquer página

% Tema
\usetheme{Madrid}

% Pacotes adicionais
\usepackage[utf8]{inputenc}
\usepackage{graphicx}
\usepackage{xcolor}
\usepackage[brazilian]{babel}
\usepackage{subcaption}
\usepackage{verbatim}
\usepackage{tabularx} % Tabelas com largura ajustável

\usepackage{xspace}  % Carrega o pacote xspace, que insere automaticamente um espaço após macros
\newcommand{\LaTeXcmd}{\LaTeX\xspace} % Define \LaTeXcmd como um atalho para \LaTeX, e usa \xspace para adicionar um espaço quando o próximo caractere for alfanumérico, evitando precisar escrever {} ou \, toda vez.
\newcommand{\TeXcmd}{\TeX\xspace} % Define \TeXcmd análogo a \LaTeXcmd, mas para o logo \TeX.

% AMBIENTE PARA CÓDIGO-FONTE
%--------------------------------------------------------
\usepackage{listings} % Permite incluir e formatar código-fonte
    \lstset{
        language=tex,               % Linguagem usada no código (aqui: LaTeX)
        basicstyle=\ttfamily\small, % Fonte monoespaçada pequena
        commentstyle=\color{gray},  % Cor dos comentários
        frame=single,               % Moldura em volta do código
        breaklines=true,            % Quebra linhas automaticamente
        captionpos=b                % Posição da legenda (b = abaixo)
    }

\renewcommand{\lstlistingname}{Código} % Muda o nome padrão de "Listing" para "Código"

    % Suporte para acentuação correta dentro do ambiente de código
    % Resolve um tipo de erro bem especifico em blocos de codigo
    \lstset{
        literate={á}{{\'a}}1 {é}{{\'e}}1 {í}{{\'i}}1 {ó}{{\'o}}1 {ú}{{\'u}}1
                 {ã}{{\~a}}1 {õ}{{\~o}}1 {â}{{\^a}}1 {ê}{{\^e}}1 {ô}{{\^o}}1
                 {À}{{\`A}}1 {à}{{\`a}}1 {ü}{{\"u}}1 {ç}{{\c{c}}}1 {Ç}{{\c{C}}}1
                 {ö}{{\"o}}1
    } % Carregando os pacotes

% --------------------------------------------------------
% ------------------------ CITAÇÕES ----------------------
% --------------------------------------------------------

% Comando para citação de abertura de capítulo com autor
\newcommand{\chapterquote}[2]{
  \begin{center}
    \small\textit{#1} \\ % A citação
    \vspace{0.5em} % Espaço entre a citação e o autor
    \small\textsc{--- #2} % O autor em small caps
  \end{center}
  \vspace{1em} % Espaçamento após a citação
}

% --------------------------------------------------------
% ------------------------ AMBIENTES ---------------------
% --------------------------------------------------------

% Contadores para cada tipo de ambiente
\newcounter{exercisecounter}
\newcounter{solutioncounter}

\newtcolorbox[auto counter]{exercisebox}[1][]{%
    beamer,
    enhanced, breakable, colback=white, % cor de fundo branca
    arc=5pt, outer arc=5pt,
    boxrule=0.5pt, colframe=black, % borda
    fontlower=\itshape,
    before=\vspace{2mm}, after=\vspace{2mm}, % Adiciona margem superior e inferior
    title={\small Questão~\thetcbcounter. #1}, % Adiciona o título com o contador e texto manual
    coltitle=white, % Define a cor do título para preto
}

\newtcolorbox[auto counter]{solutionbox}[1][]{%
	beamer,
    enhanced, breakable, colback=gray!10, colframe=gray!10,
    arc=10pt, outer arc=10pt,
    boxrule=0pt,
    before=\vspace{2mm}, after=\vspace{2mm},
    title={\small \textsc{Solução~\thetcbcounter}},
    coltitle=black, % Adiciona o título com o contador
    #1
}

\newtcolorbox{highlightbox}[1][]{%
	beamer,
    enhanced, breakable, colback=gray!10, colframe=gray!10, % cor de fundo atualizada para gray!10
    arc=0pt, outer arc=0pt,
    boxrule=0pt,
    before=\vspace{2mm}, after=\vspace{2mm},
    fonttitle=\bfseries\scshape, % Define o título em negrito e small caps
    coltitle=black, % Define a cor do título como preta
    #1}

% --------------------------------------------------------
% ------------------------ SUMÁRIO ---------------------
% --------------------------------------------------------

\addto\captionsportuguese{
  \renewcommand{\contentsname}{\hfill\bfseries\huge Sumário \hfill}   
}
\renewcommand{\cfttoctitlefont}{\hfill\bfseries\huge}
\renewcommand{\cftaftertoctitle}{\hfill\break\vspace{10pt}}

% Define o símbolo de quadrado preenchido
\newcommand{\filledsquare}{{\tikz{\node[fill=black,rectangle,minimum size=0.2cm] {};}}\hspace{0.5em}}

% Títulos de capítulos em negrito no sumário com o símbolo
\renewcommand{\cftchapfont}{\bfseries\large\filledsquare}

\begin{document}

% Primeira página com a imagem como background
\AddToShipoutPictureBG*{\includegraphics[width=\paperwidth,height=\paperheight]{capa.png}} % Substitua "capa.png" pelo nome real do arquivo da imagem
\begin{titlepage}
    \null % Adiciona um espaço vazio para que a imagem possa ser exibida corretamente
\end{titlepage}
\ClearShipoutPictureBG % Limpa o background da próxima página

\clearpage % Começa uma nova página para o conteúdo do livro

\thispagestyle{empty} % No page number on this page
\centering

\vspace{1cm} % Ajuste o espaço conforme necessário

{\Large\bfseries\textsc{\MakeUppercase{The Overleafify Project}}}\\[1cm]

{\large\textsc{Felipe Silva \& Yuri Thadeu}}\\
{IEEE Student Branch USP São Carlos}\\[20mm]  
\flushleft
\begin{minipage}{\textwidth}

\textsc{Resumo.} Este livro é um guia prático e acessível para iniciantes no uso do \LaTeX{} e do Overleaf, ferramentas essenciais para a produção de documentos acadêmicos e técnicos de alta qualidade. Abordamos desde os conceitos básicos, como a estrutura de documentos e formatação de texto, até tópicos avançados, incluindo a criação de tabelas, inserção de imagens, expressões matemáticas, bibliografias e gráficos. Além disso, exploramos a criação de apresentações profissionais com o Beamer e a personalização de layouts para atender às necessidades específicas de cada projeto. Com exemplos práticos, exercícios e dicas úteis, este guia visa capacitar os novos ingressantes a dominar essas ferramentas de forma eficiente, tornando a documentação científica mais clara, organizada e esteticamente atraente.
\end{minipage}\\[1cm]   
\vfill
    \begin{center}
        Edição: 1.0  \\
    	\textit{Copyright \copyright 2025 \\ IEEE Student Branch USP São Carlos.}
    \end{center}
    \begin{center}
    	\textit{Este livro é distribuído sob uma licença aberta. A cópia, reprodução e compartilhamento são permitidos, desde que sejam mantidos os créditos aos autores.}
    \end{center}
\clearpage % Assegura que a próxima seção comece em uma nova página

\setcounter{page}{1} % Reseta o contador de páginas para o conteúdo principal

\thispagestyle{empty}
\tableofcontents

\pagestyle{empty} % Remove cabeçalho e rodapé

\chapter*{Prefácio}
\justify
\addcontentsline{toc}{chapter}{Prefácio}

Este livro surgiu como um projeto extracurricular, fruto do nosso interesse em documentação acadêmica e, especialmente, na utilização do \LaTeX{} como ferramenta para a elaboração de textos técnicos e científicos. Nossa motivação inicial foi criar um material que facilitasse o aprendizado dessa linguagem, proporcionando um guia acessível e progressivo para estudantes e pesquisadores.

Ao longo de nossas trajetórias acadêmicas, enfrentamos desafios comuns que muitos iniciantes no \LaTeX{} também encontram: desde a configuração inicial até a compreensão de suas funcionalidades avançadas. Muitas vezes, a falta de material introdutório estruturado dificulta o aprendizado, tornando o processo mais demorado e desmotivador. Essa experiência nos motivou a desenvolver este livro, buscando oferecer um caminho mais fluido para aqueles que desejam dominar o \LaTeX{} de forma prática e objetiva.

Diferente de tutoriais fragmentados encontrados na internet, este livro busca fornecer uma abordagem sequencial e bem estruturada, cobrindo desde os fundamentos básicos até tópicos avançados. Esperamos que ele auxilie estudantes e pesquisadores a superar as dificuldades iniciais e explorar ao máximo o potencial do \LaTeX{} na escrita acadêmica e técnica.

\clearpage
\pagestyle{plain} % Restaura o estilo de página padrão para o restante do documento

\pagestyle{empty} % Remove cabeçalho e rodapé

\chapter*{Introdução}
\justify
\addcontentsline{toc}{chapter}{Introdução}

A escrita acadêmica exige documentos bem estruturados, claros e organizados, especialmente em publicações científicas, teses e relatórios. Ferramentas convencionais de edição de texto nem sempre oferecem a precisão e o controle necessários para lidar com referências, equações matemáticas e formatação avançada.

O \LaTeX{} surgiu como uma solução para esses desafios, permitindo a produção de documentos com qualidade tipográfica superior, automação de citações e numeração eficiente de figuras e tabelas. Hoje, ele é amplamente adotado em diversas áreas acadêmicas e tornou-se um padrão para publicações técnicas e científicas.

\section*{\TeX{} vs \LaTeX{}: qual a diferença?}

Antes de aprofundarmos o estudo do \LaTeX{}, é importante entender sua origem. O \textbf{\TeX{}} foi criado por Donald Knuth na década de 1970 com o objetivo de fornecer um sistema preciso e de alta qualidade para composição tipográfica de textos científicos e matemáticos. No entanto, por ser uma ferramenta de \textit{baixo nível}, seu uso exigia um grande conhecimento técnico.

O \textbf{\LaTeX{}} desenvolvido por Leslie Lamport nos anos 1980, é uma camada de alto nível sobre o \TeX{}, que simplifica seu uso e estrutura os documentos de maneira mais intuitiva. Com o \LaTeX{}, o usuário pode se concentrar no \textit{conteúdo}, enquanto o software se encarrega da formatação.

\section*{Por que \LaTeX{} e não Word ou outras ferramentas?}

Muitos usuários iniciantes questionam por que aprender \LaTeX{} se existem ferramentas como Microsoft Word e Google Docs, que possuem interfaces gráficas mais amigáveis. A resposta está na diferença fundamental entre dois modelos de edição de texto:

\begin{itemize}
  \item \textbf{WYSIWYG (What You See Is What You Get)}: Modelos como Word e Google Docs seguem esse princípio, onde o usuário edita o documento visualmente e a formatação é ajustada manualmente.
  \item \textbf{WYSIWYM (What You See Is What You Mean)}: O \LaTeX{} segue esse paradigma, onde o usuário define a estrutura e o conteúdo, e a formatação é gerenciada automaticamente pelo sistema.
\end{itemize}

\subsection*{Por que o WYSIWYM do \LaTeX{} é uma melhor escolha?}

A principal vantagem do modelo WYSIWYM é a separação entre o conteúdo e a aparência do documento. No Word, por exemplo, o usuário precisa constantemente ajustar margens, fontes, espaçamentos e outros elementos visuais. Isso pode ser um problema em documentos longos, pois qualquer alteração pode desorganizar a estrutura do texto.

No \LaTeX{}, a formatação é definida de maneira centralizada e padronizada. Isso significa que um documento pode ser reformatado inteiramente apenas mudando algumas configurações no preâmbulo, sem a necessidade de ajustar cada elemento manualmente. Essa abordagem garante mais consistência e evita erros comuns de formatação.

Outro ponto relevante é a produtividade. Como o usuário escreve o conteúdo sem se preocupar com a estética, o fluxo de escrita se torna mais eficiente. Além disso, recursos como referências automáticas, numeração correta de figuras e fórmulas matemáticas integradas tornam o \LaTeX{} a melhor opção para trabalhos acadêmicos e técnicos.

\subsection*{Benefícios do \LaTeX{}, sobre editores tradicionais}

\begin{itemize}
  \item \textbf{Consistência tipográfica}: O LaTeX garante que o documento tenha um estilo uniforme, sem necessidade de ajustes manuais de formatação.
  \item \textbf{Gerenciamento automático de referências e bibliografia}: Ferramentas como BibTeX permitem organizar e citar fontes de maneira automatizada.
  \item \textbf{Numeração automática de figuras, tabelas e equações}: Diferente do Word, onde esses elementos precisam ser ajustados manualmente.
  \item \textbf{Alta qualidade tipográfica}: Textos em \LaTeX{} possuem uma apresentação mais profissional.
  \item \textbf{Suporte nativo a fórmulas matemáticas}: Essencial para áreas como Matemática, Engenharia e Física.
\end{itemize}

\section*{\LaTeX{} na comunidade acadêmica e profissional}

O \LaTeX{} não é usado apenas por acadêmicos; ele se tornou um padrão na produção de documentos em diversas áreas:

\begin{itemize}
  \item \textbf{Ciências Exatas e Engenharia}: Matemáticos, físicos e engenheiros utilizam o \LaTeX{} para escrever artigos científicos, \textit{papers} e teses.
  \item \textbf{Publicação Científica}: Revistas como IEEE, ACM e Springer exigem que os artigos sejam submetidos em \LaTeX{}.
  \item \textbf{Economia e Ciências Sociais}: Pesquisadores dessas áreas utilizam \LaTeX \,para artigos e relatórios devido à sua eficiência na formatação e referência cruzada.
  \item \textbf{Publicação de Livros e Relatórios Técnicos}: Editoras utilizam o \LaTeX \,para garantir formatação padronizada e profissional.
\end{itemize}

Além disso, repositórios acadêmicos como \textbf{arXiv} e \textbf{Overleaf} popularizaram ainda mais seu uso, permitindo a colaboração em tempo real na produção de documentos.

\section*{O ambiente de trabalho no \LaTeX{}}

O \LaTeX{} pode ser utilizado em diferentes ambientes, dependendo das preferências do usuário:

\begin{itemize}
  \item \textbf{Compiladores Offline}: Softwares como MikTeX e TeX Live permitem compilar documentos no próprio computador.
  \item \textbf{Plataformas Online}: Serviços como Overleaf oferecem uma solução baseada na nuvem, permitindo colaboração e edição simultânea.
\end{itemize}

Neste livro, nosso foco será no uso do \textbf{Overleaf}, uma das plataformas mais acessíveis e fáceis de utilizar. O Overleaf elimina a necessidade de instalação local, oferece templates prontos e facilita a compilação de documentos \LaTeX{} sem necessidade de configurações complexas.

\section*{Mitos e verdades sobre o \LaTeX{}}

Ao longo dos anos, o \LaTeX{} colecionou certa aura de “ferramenta para iniciados”, o que acaba alimentando dúvidas e mitos entre novos usuários. Desvendamos aqui os equívocos mais comuns e explicamos por que, mesmo exigindo um investimento inicial de tempo, o \LaTeX{} continua valendo a pena para quem produz documentos técnicos e acadêmicos.

\subsection*{Você nunca vai “saber tudo” — e tudo bem}

O ecossistema do \LaTeX{} é construído sobre \emph{packages}. Há milhares delas, voltadas a necessidades ultra‑específicas — de diagramas de química orgânica a partituras de música barroca. Esperar dominar cada detalhe é irreal; na prática, você vai conhecer profundamente o conjunto de pacotes que usa com frequência e manter uma lista de referência para o resto. A flexibilidade que isso oferece supera a aparente complexidade inicial.

\subsection*{Curva de aprendizado: quanto tempo, afinal?}

Aprender \LaTeX{} exige, sim, um investimento inicial: é preciso se familiarizar com a sintaxe, o fluxo de compilação e alguns pacotes essenciais. No entanto, esse esforço logo se converte em ganho de produtividade, pois o sistema passa a cuidar da formatação enquanto você se concentra apenas no conteúdo.

\begin{itemize}
  \item No início, foque em compreender a estrutura básica de um documento e em compilar exemplos simples.
  \item Com a prática e a repetição, os comandos tornam‑se naturais, e relatórios completos são produzidos sem ajustes manuais de formatação.
\end{itemize}

Em suma, o tempo dedicado hoje se paga rapidamente no futuro: à medida que você escreve mais, o processo se torna cada vez mais ágil e consistente — especialmente em projetos longos, nos quais a automação e a padronização fazem toda a diferença.

\subsection*{O investimento se paga — e não demora tanto}

Se por um lado a configuração inicial parece trabalhosa, por outro cada minuto dedicado se converte em horas economizadas em revisões futuras. Quem já perdeu um dia inteiro ajustando margens quebradas no Word costuma migrar para o \LaTeX{} depois de uma “crise de formatação”. Uma vez dominado o fluxo, tarefas como reestruturar capítulos ou padronizar citações viram comandos de poucos caracteres.

\subsection*{Armadilhas de produtividade (e como evitá‑las)}

O \LaTeX{} pode virar um campo minado de \emph{tweaks} estéticos: trocar fonte, alinhar vírgulas ou redesenhar a capa pode consumir tardes inteiras. Para fugir desse ciclo:
\begin{enumerate}
  \item Adote um \emph{template} estável e altere‑o só quando houver necessidade real.
  \item Separe sessões de “escrita” e de “tipografia”; nunca as misture.
  \item Mantenha um arquivo \texttt{.tex} com exemplos mínimos dos truques que você mais esquece e exemplos que você mais usa.
\end{enumerate}

\subsection*{Dicas para acelerar o aprendizado}

\begin{itemize}
  \item Comece por projetos pequenos (resumos, relatórios curtos) antes de encarar a sua dissertação.
  \item Use a função \emph{auto‑complete} do Overleaf e escolha um tema do editor que te agrade.
  \item Consulte comunidades online sempre que encontrar erros muito específicos.
\end{itemize}

\vfill
\begin{center}
  \begin{quote}
    \emph{Aprender \LaTeX{} não é um evento pontual, mas um processo contínuo. A boa notícia é que você começa a colher os frutos muito antes de “chegar ao final” — porque, na verdade, não há final.}
  \end{quote}
\end{center}

\clearpage
\pagestyle{plain} % Restaura o estilo de página padrão para o restante do documento

\include{capitulos/1_typer_doc_struct}
\chapter{Texto e formatação}
\label{cap:2_text_formatacao}

\chapterquote{``A mente que se abre a uma nova ideia jamais voltará ao seu tamanho original.''}{Albert Einstein}

A formatação de texto no \LaTeX{} permite a criação de documentos com estrutura profissional e padronizada. Recursos como negrito, itálico, sublinhado e espaçamento adequado são essenciais para organizar o conteúdo e destacar informações importantes. Segundo Mittelbach et al., a estruturação do texto no \LaTeX{} segue rigorosos padrões tipográficos que garantem clareza e precisão \cite{mittelbach2004latex}.


O \LaTeX{} permite uma formatação de texto estruturada, facilitando a organização de documentos acadêmicos \cite{mittelbach2004latex}.

% --------

\section{Formatação básica}
\label{sec:cap2_formatação_básica} % Referência cruzada para esta seção

\begin{itemize}
    \item \textbf{Negrito}: \verb|\textbf{Texto em negrito}|
    \item \textit{Itálico}: \verb|\textit{Texto em itálico}|
    \item \underline{Sublinhado}: \verb|\underline{Texto sublinhado}|
\end{itemize}

\begin{lstlisting}[language=tex, caption=Formatação básica de texto]
    \textbf{Importante}: \textit{Não esqueça} de \underline{revisar} o código.
\end{lstlisting}

O código acima gera como saída:\\

\textbf{Importante}: \textit{Não esqueça} de \underline{revisar} o código.

\section{Tamanhos da fonte}

Tamanhos pré-definidos (do menor ao maior):
{\tiny tiny}  
{\scriptsize scriptsize}  
{\footnotesize footnotesize}  
{\small small}  
{\normalsize normalsize}  
{\large large}  
{\Large Large}  
{\LARGE LARGE}  
{\huge huge}  
{\Huge Huge}

\begin{lstlisting}[language=tex, caption=Tamanhos de fontes pré-definidos]
    {\tiny Texto pequeno}  
    {\scriptsize Um pouco maior}  
    {\footnotesize Notas de rodapé}  
    {\small Pequeno}  
    {\normalsize Normal}  
    {\large Grande}  
    {\Large Maior}  
    {\LARGE Muito grande}  
    {\huge Enorme}  
    {\Huge Gigante}  
\end{lstlisting}

Dica: Use a forma com \{…\} sempre que quiser limitar o efeito do comando de tamanho a um trecho:

\begin{lstlisting}[language=tex, caption=Tamanhos de fontes pré-definidos aplicado apenas no trecho entre chaves]
    {\large Este texto é apenas grande aqui}
\end{lstlisting}

Ele volta automaticamente ao padrão ao fechar as chaves.\\

Se você usar o comando como declaração solta:

\begin{lstlisting}[language=tex, caption=Tamanho de fonte como declaração solta]
    \large
    Texto fica grande até aqui.
    \normalsize
\end{lstlisting}

é preciso chamar \verb|\normalsize| para retomar o tamanho normal após o bloco.\\

Em suma:

\begin{lstlisting}[language=tex, caption=Resumo sobre a dinâmica de tamanho de fontes]
    \documentclass{article}
    \begin{document}
    
    % 1) Com escopo em chaves: volta sozinho
    Antes {\huge Muito grande} Depois\\
    
    % 2) Declaração solta: precisa de \normalsize
    \huge
    Este parágrafo fica grande até aqui.\\
    
    \normalsize % volta ao tamanho "normal"
    Agora está normal de novo.\\
    
    \end{document}
\end{lstlisting}

% 1) Com escopo em chaves: volta sozinho
Antes {\huge Muito grande} Depois\\

% 2) Declaração solta: precisa de \normalsize
\huge
Este parágrafo fica grande até aqui.\\

\normalsize % volta ao tamanho "normal"
Agora está normal de novo.\\

\section{Cores no texto}

Use o pacote \verb|xcolor|:

\begin{lstlisting}[language=tex, caption=Uso do pacote \texttt{xcolor} para customização de cores no texto]
    \usepackage{xcolor}  
    \textcolor{red}{Texto vermelho}  
    \textcolor{blue}{Texto azul}  
    \textcolor[HTML]{00FF00}{Verde em hexadecimal}  
\end{lstlisting}

\textcolor{red}{Texto vermelho}  
\textcolor{blue}{Texto azul}  
\textcolor[HTML]{00FF00}{Verde em hexadecimal}  

\section{Listas}

\subsection{Listas enumeradas - ambiente \texttt{enumerate}}

\begin{lstlisting}[language=tex, caption=Listas enumeradas]
    \begin{enumerate}
        \item Primeiro item
        \item Segundo item
    \end{enumerate} 
\end{lstlisting}

\begin{enumerate}
    \item Primeiro item
    \item Segundo item
\end{enumerate}

\subsection{Listas não enumeradas - ambiente \texttt{itemize}}

\begin{lstlisting}[language=tex, caption=Listas não enumeradas]
    \begin{itemize}
        \item Item com marcador padrão
        \item Outro item
    \end{itemize}
\end{lstlisting}

\begin{itemize}
    \item Item com marcador padrão
    \item Outro item
\end{itemize}

\subsection{Listas descritivas - ambiente \texttt{description}}

\begin{lstlisting}[language=tex, caption=Listas descritivas]
    \begin{description}
        \item[\LaTeX] Sistema de preparação de documentos.
        \item[Overleaf] Plataforma online para LaTeX.
    \end{description}
\end{lstlisting}

\begin{description}
    \item[\LaTeX] Sistema de preparação de documentos.
    \item[Overleaf] Plataforma online para \LaTeX{}.
\end{description}

Dica: Use o pacote \verb|enumitem| para personalizar marcadores e espaçamento.

\section{Espaçamento e parágrafos}
\begin{itemize}
    \item Quebra de linha: \verb|\\| ou \verb|\newline|
    \item Novo parágrafo: Deixe uma linha vazia no código ou use \verb|\par|.
    \item Espaçamento vertical: \verb|\vspace{1cm}|
    \item Espaçamento horizontal: \verb|\hspace{2em}|
\end{itemize}

\begin{lstlisting}[language=tex, caption=Exemplo de espaçamento e parágrafos]
    Primeira linha.\\ Segunda linha.  
    \par Terceira linha (novo parágrafo).
\end{lstlisting}

Primeira linha.\\ Segunda linha.  
\par Terceira linha (novo parágrafo).

% \section{Espaçamento entre linhas}

% Use o pacote \verb|setspace|:

\section{Dicas práticas}

\begin{itemize}
    \item Pacotes úteis:
    \begin{lstlisting}[language=tex, caption=Outros pacotes úteis para texto e formatação]
    \usepackage{lipsum}   % Gerar texto aleatório para testes
    \usepackage{hyperref} % Links clicáveis
    \end{lstlisting}
    \item Erros comuns:
    \begin{itemize}
        \item Esquecer de fechar chaves \verb|{}| após comandos de formatação.
        \item Usar \verb|\\| excessivamente (pode causar layout quebrado).
    \end{itemize}
\end{itemize}

\bigskip
\noindent\textbf{Transição para o próximo capítulo:}\\[1ex]
Agora que você já dominou os conceitos básicos apresentados neste capítulo, no Capítulo~\ref{cap:3_tabelas} vamos colocar em prática a formatação de tabelas. Você aprenderá a criar, ajustar e personalizar tabelas para organizar dados de forma clara e eficiente – um recurso essencial para documentos bem estruturados.



\chapter{Criação de tabelas}
\label{cap:3_tabelas}

\chapterquote{``Acima de tudo, mostre os dados.''}{Edward R. Tufte}

Tabelas são elementos essenciais para a organização e apresentação de dados em documentos técnicos e científicos. O \LaTeX{} oferece um sistema flexível para criar tabelas com controle preciso sobre alinhamento, bordas, espaçamento e personalização. Este capítulo aborda a sintaxe do ambiente \texttt{tabular}, explicando como estruturar e formatar tabelas de maneira eficiente.

% --------

\section{Tabelas básicas com \texttt{tabular}}
\label{sec:cap3_tabelas_básicas_com_tabular} % Referência cruzada para esta seção

O ambiente \verb|tabular| é a base para criar tabelas em \LaTeX{}, oferecendo controle sobre alinhamento e espaçamento \cite{mittelbach2004latex}.

\begin{lstlisting}[language=tex, caption=Sintaxe básica para criação de tabelas]
    \begin{tabular}{colunas}
        Conteúdo das células & separado por & \\
        Linhas separadas por \\ 
    \end{tabular}
\end{lstlisting}

\newpage

\begin{lstlisting}[language=tex, caption=Exemplo básico de tabela]
    \begin{center}
        \begin{tabular}{|c|c|c|}  % Colunas centralizadas (c) com bordas (|)
            \hline
            Nome & Idade & Cidade \\ 
            \hline
            João & 25 & São Paulo \\ 
            Maria & 30 & Rio de Janeiro \\ 
            \hline
        \end{tabular}
    \end{center}
\end{lstlisting}

\begin{center}
    \begin{tabular}{|c|c|c|}  % Colunas centralizadas (c) com bordas (|)
        \hline
        Nome & Idade & Cidade \\ 
        \hline
        João & 25 & São Paulo \\ 
        Maria & 30 & Rio de Janeiro \\ 
        \hline
    \end{tabular}
\end{center}

\section{Tipos de colunas}

\begin{itemize}
    \item \verb|c|: Centralizada.
    \item \verb|l|: Alinhada à esquerda.
    \item \verb|r|: Alinhada à direita;
    \item \verb|||: Adiciona bordas verticais
    \item \verb|p{largura}|: Coluna com largura fixa
\end{itemize}

\begin{lstlisting}[language=tex, caption=Exemplo com alinhamento]
    \begin{tabular}{l r p{4cm}}
        Nome (esquerda) & Preço (direita) & Descrição (largura fixa) \\ 
        \hline
        Livro & R\$ 50,00 & Um livro sobre LaTeX \\ 
        Caneta & R\$ 2,50 & Caneta azul \\ 
    \end{tabular}
\end{lstlisting}

\begin{tabular}{l r p{4cm}}
    Nome (esquerda) & Preço (direita) & Descrição (largura fixa) \\ 
    \hline
    Livro & R\$ 50,00 & Um livro sobre \LaTeX{} \\ 
    Caneta & R\$ 2,50 & Caneta azul \\ 
\end{tabular}

\section{Pacote \texttt{tabularx} para largura ajustável}

Permite criar tabelas que se ajustam à largura do texto.

\begin{lstlisting}[language=tex, caption=Uso do pacote \texttt{tabularx}]
\begin{tabularx}{\textwidth}{|X|X|X|} % Colunas X dividem o espaço igualmente
    \hline
    Cabeçalho 1 & Cabeçalho 2 & Cabeçalho 3 \\ 
    \hline
    Texto longo que se ajusta automaticamente & Dado 2 & Dado 3 \\ 
    \hline
\end{tabularx}
\end{lstlisting}


\begin{tabularx}{\textwidth}{|X|X|X|} % Colunas X dividem o espaço igualmente
    \hline
    Cabeçalho 1 & Cabeçalho 2 & Cabeçalho 3 \\ 
    \hline
    Texto longo que se ajusta automaticamente & Dado 2 & Dado 3 \\ 
    \hline
\end{tabularx}


\section{Linhas horizontais e verticais}

\begin{itemize}
    \item \verb|\hline|: Linha horizontal.
    \item \verb|cline{i-j}|: Linha horizontal parcial.
    \item \verb|||: Bordas verticais (ex: \texttt{|c|c|c|}).
\end{itemize}

\begin{lstlisting}[language=tex, caption=Exemplo de criação de tabela]
    \begin{center}
        \begin{tabular}{|l|c|r|}
            \hline
            \multicolumn{3}{|c|}{Título da Tabela} \\ 
            \hline
            Item & Quantidade & Preço \\ 
            \cline{1-2} % Linha parcial
            Livro & 2 & R\$ 100,00 \\ 
            \hline
        \end{tabular}
    \end{center}
\end{lstlisting}

\begin{center}
    \begin{tabular}{|l|c|r|}
        \hline
        \multicolumn{3}{|c|}{Título da Tabela} \\ 
        \hline
        Item & Quantidade & Preço \\ 
        \cline{1-2} % Linha parcial
        Livro & 2 & R\$ 100,00 \\ 
        \hline
    \end{tabular}
\end{center}

\section{Mesclando células}

\begin{itemize}
    \item \verb|\multicolumn{n}{alinhamento}{texto}|: Mescla colunas.
    \item \verb|\multirow{n}{largura}{texto}|: Mescla linhas (requer o pacote \verb|multirow|).
\end{itemize}

\begin{lstlisting}[language=tex, caption=Exemplo de mesclagem]
\begin{center}
    \begin{tabular}{|c|c|c|}
        \hline
        \multirow{2}{*}{Categoria} & \multicolumn{2}{c|}{Dados} \\ 
        \cline{2-3}
        & Produto & Preço \\ 
        \hline
        Livros & LaTeX Guide & R\$ 80,00 \\ 
        \hline
    \end{tabular}
\end{center}
\end{lstlisting}

\begin{center}
    \begin{tabular}{|c|c|c|}
        \hline
        \multirow{2}{*}{Categoria} & \multicolumn{2}{c|}{Dados} \\ 
        \cline{2-3}
        & Produto & Preço \\ 
        \hline
        Livros & LaTeX Guide & R\$ 80,00 \\ 
        \hline
    \end{tabular}
\end{center}

\section{Tabelas profissionais com \texttt{booktabs}}

O pacote \verb|booktabs| remove bordas excessivas e melhora a estética:

\begin{lstlisting}[language=tex, caption=Exemplo de mesclagem]
\begin{center}
    \begin{tabular}{lcc}
        \toprule
        Nome & Nota 1 & Nota 2 \\ 
        \midrule
        João & 8,5 & 9,0 \\ 
        Maria & 7,0 & 8,5 \\ 
        \bottomrule
    \end{tabular}
\end{center}
\end{lstlisting}

\begin{center}
    \begin{tabular}{lcc}
        \toprule
        Nome & Nota 1 & Nota 2 \\ 
        \midrule
        João & 8,5 & 9,0 \\ 
        Maria & 7,0 & 8,5 \\ 
        \bottomrule
    \end{tabular}
\end{center}

Dica: Evite usar \verb|\hline| com \verb|booktabs| para manter o estilo limpo.

\section{Dicas práticas}

\begin{itemize}
    \item Pacotes úteis:
    \begin{lstlisting}[language=tex, caption=Pacotes úteis e interessantes para usar com tabelas]
    \usepackage{array}       % Mais controle sobre colunas
    \usepackage{graphicx}    % Inserir imagens em células
    \end{lstlisting}
    \item Erros comuns:
    \begin{itemize}
        \item Esquecer de adicionar \& entre células.
        \item Usar \verb|\\| no final da última linha (desnecessário).
    \end{itemize}
\end{itemize}

\bigskip
\noindent\textbf{Transição para o próximo capítulo:}\\[1ex]
Agora que você aprendeu a organizar e formatar tabelas de forma clara e elegante, no Capítulo~\ref{cap:4_manip_img} vamos mergulhar na Manipulação de Imagens em \LaTeX{}. Você descobrirá como inserir, ajustar e posicionar imagens de modo que elas complementem e enriqueçam seus documentos.
    

\chapter{Inserção e ajuste de Imagens em LaTeX}\label{cap:4_manip_img}

\chapterquote{``Uma imagem vale mais que mil palavras.''}{Fred R. Barnard}

A inserção de imagens é um aspecto fundamental na produção de documentos acadêmicos e técnicos. O \LaTeX{} permite manipular gráficos e figuras de maneira precisa, garantindo um posicionamento adequado e compatibilidade com diferentes formatos. Este capítulo apresenta o uso do pacote \verb|graphicx|, explicando como importar, redimensionar e posicionar imagens corretamente.

Caso precise gerar gráficos diretamente no \LaTeX{}, veja o \autoref{cap:8_graf_diagra}, que apresenta os pacotes \verb|TikZ| e \verb|pgfplots|.

% --------

\section{Introdução ao pacote \texttt{graphicx}}
\label{sec:cap4_introdução_ao_pacote_graphicx} % Referência cruzada para esta seção

O pacote \verb|graphicx| é essencial para inserir imagens \cite{overleaf2022guide}. Carregue-o no pré-âmbulo:

\begin{lstlisting}[language=tex, caption=Carregue o pacote \texttt{graphicx}]
    \usepackage{graphicx} % Sempre adicione isso!
\end{lstlisting}

Formato de imagens suportados:
\begin{itemize}
    \item \textbf{Overleaf}: \verb|PNG|, \verb|JPG|, \verb|PDF| (evite \verb|SVG|, a menos que convertido.)
    \item \textbf{LaTeX offline}: \verb|PDF|, \verb|EPS|, \verb|PNG|, \verb|JPG|
\end{itemize}

\section{Inserindo imagens básicas}

Use \verb|\includegraphics| para adicionar uma imagem:

\begin{lstlisting}[language=tex, caption=Inserindo uma imagem]
    \includegraphics[width=0.5\textwidth]{caminho/para/imagem.png}  
\end{lstlisting}

\section{Redimensionamento e escala}

\begin{itemize}
    \item Largura fixa: \verb|width=8cm|.
    \item Altura fixa: \verb|height=4cm|.
    \item Escala proporcional: \verb|scale=0.7|.
    \item Rotação: \verb|angle=45|
\end{itemize}

\begin{lstlisting}[language=tex, caption=Exemplo de customização do tamanho e ângulo de imagem]
    \includegraphics[width=5cm, angle=30]{imagem.jpg}  
\end{lstlisting}


\section{Posicionamento com o ambiente \texttt{figure}}

O ambiente \verb|figure| permite controle profissional sobre posição, legendas e referências:

\begin{lstlisting}[language=tex, caption=Exemplo de customização do tamanho e ângulo de imagem]
    \begin{figure}[h!] % Opções: h (aqui), t (topo), b (base), p (página separada)
        \centering
        \includegraphics[width=0.4\textwidth]{imagens/sapo.jpg}
        \caption{Imagem da capa do livro} % Legenda
        \label{fig:sapodacapa} % Rótulo para referência
    \end{figure}
\end{lstlisting}

\begin{figure}[h!] % Opções: h (aqui), t (topo), b (base), p (página separada)
    \centering
    \includegraphics[width=0.7\textwidth]{imagens/sapo.jpg}
    \caption{Imagem da capa do livro} % Legenda
    \label{fig:sapodacapa} % Rótulo para referência
\end{figure}

\section{Múltiplas imagens em uma figura}

Use o pacote \verb|subcaption| para criar subfiguras:

\begin{lstlisting}[language=tex, caption=Exemplo de customização do tamanho e ângulo de imagem]
\begin{figure}[h]
    \centering
    \begin{subfigure}{0.45\textwidth}
        \includegraphics[width=\textwidth]{imagem1.jpg}
        \caption{Primeira imagem}
    \end{subfigure}
    \hfill
    \begin{subfigure}{0.45\textwidth}
        \includegraphics[width=\textwidth]{imagem2.jpg}
        \caption{Segunda imagem}
    \end{subfigure}
    \caption{Conjunto de imagens}
    \label{fig:duas-imagens}
\end{figure}
\end{lstlisting}

\begin{figure}[h]
    \centering
    \begin{subfigure}{0.45\textwidth}
        \includegraphics[width=\textwidth]{imagens/sapo1.jpg}
        \caption{Primeira imagem}
    \end{subfigure}
    \hfill
    \begin{subfigure}{0.45\textwidth}
        \includegraphics[width=\textwidth]{imagens/sapo2.jpg}
        \caption{Segunda imagem}
    \end{subfigure}
    \caption{Conjunto de imagens}
    \label{fig:duas-imagens}
\end{figure}

\newpage

\section{Dicas práticas}
\begin{itemize}
    \item Armazene as imagens em uma pasta chamada \verb|imagens| e use:
    \begin{lstlisting}[language=tex, caption=Exemplo de customização do tamanho e ângulo de imagem]
    \graphicspath{{imagens/}} % Define o caminho padrão
    \includegraphics{logo} % Busca por "logo.png" ou "logo.jpg"
    \end{lstlisting}
    \item Erros comuns:
    \begin{itemize}
        \item Imagem não aparece: Verifique o caminho do arquivo e certifique que o pacote \verb|graphicx| está carregado.
        \item Legenda fora do lugar: Use \verb|\centering| dentro do ambiente \verb|figure|.
        \item Formato não suportado: Converta a imagem para um dos formatos suportados. 
    \end{itemize}
    
\end{itemize}

\bigskip
\noindent\textbf{Transição para o próximo capítulo:}\\[1ex]
Após aprender a inserir e ajustar imagens para enriquecer seus documentos, no Capítulo~\ref{cap:5_Exp_matematica} você explorará Expressões Matemáticas, onde serão abordadas a formatação e a inserção de equações e símbolos essenciais para a elaboração de textos acadêmicos.

\chapter{Expressões matemáticas}\label{cap:5_Exp_matematica}

\chapterquote{``A matemática é o alfabeto com o qual Deus escreveu o universo.''}{Galileo Galilei}

A escrita de expressões matemáticas no \LaTeX{} proporciona um alto nível de precisão e clareza \cite{knuth1984texbook} na apresentação de equações e fórmulas. Com suporte nativo a operadores, símbolos e estruturas complexas, o \LaTeX{} é amplamente utilizado em publicações científicas e acadêmicas. Este capítulo explora os diferentes modos matemáticos e os comandos essenciais para a formatação de expressões matemáticas.

% --------

\section{Modos matemáticos}
\label{sec:cap5_modos_matemáticos} % Referência cruzada para esta seção

\begin{itemize}
    \item Inline (dentro do texto): Use \verb|\(...\)| ou \verb|$...$|:
    \begin{lstlisting}[language=tex, caption=Equação inline]
    A equação \(E = mc^2\) foi proposta por Einstein.
    \end{lstlisting}
    Saída: A equação \(E = mc^2\) foi proposta por Einstein.
    \item Display: Use \verb|\[...\]| ou \verb|$$...$$|:
    \begin{lstlisting}[language=tex, caption=Equação display]
    A fórmula da energia é:
    \[E = mc^2\]
    \end{lstlisting}
    Saída: \[E = mc^2\]
\end{itemize}

\section{Símbolos básicos}

\begin{itemize}
    \item Letras gregas: \verb|\alpha|, \verb|\beta|, \verb|\gamma|, \verb|\Gamma|, \verb|\Delta|, \verb|\Omega|
    \(\alpha\), \(\beta\), \(\gamma\), \(\Gamma\), \(\Delta\), \(\Omega\)
    \item Operadores e relações: \verb|\times| (multiplicação), \verb|\pm| (mais/menos), \verb|\leq| (menor ou igual), \verb|\geq| (maior ou igual)
    \(\times\), \(\pm\), \(\leq\), \(\geq\)
    \item Setas e símbolos lógicos: \verb|\leftarrow|, \verb|\Leftarrow|, \verb|\rightarrow|, \verb|\Rightarrow|, \verb|\exists| (existe), \verb|\forall| (para todo)
    \(\leftarrow\), \(\Leftarrow\), \(\rightarrow\), \(\Rightarrow\), \(\exists\), \(\forall\)
\end{itemize}

Para mais símbolos matemáticos, vide a tabela \ref{tab:simbolos} no apêndice.

\section{Frações, expoentes e índices}

\begin{itemize}
    \item Frações:
    \begin{lstlisting}[language=tex, caption=Frações em LaTeX]
    \begin{center}
        \(\frac{a}{b}\)\\
    \end{center}
    \[\dfrac{a}{b}\] (uso em display mode)
    \end{lstlisting}
    \begin{center}
        \(\frac{a}{b}\)\\
    \end{center}
    \[\dfrac{a}{b}\]
\end{itemize}

Use \verb|dfrac{a}{b}| para o modo display - o tamanho fica maior.

\section{Matrizes e vetores}

\begin{itemize}
    \item Matrizes simples:
    \begin{lstlisting}[language=tex, caption=Matriz simples]
    \[
    \begin{matrix}
    1 & 2 \\
    3 & 4
    \end{matrix}
    \]
    \end{lstlisting}
    \[
    \begin{matrix}
    1 & 2 \\
    3 & 4
    \end{matrix}
    \]

    \item Matrizes com parênteses (usando \verb|pmatrix|):
    \begin{lstlisting}[language=tex, caption=Matriz com parênteses]
    \[
    \begin{pmatrix}
    a & b \\
    c & d
    \end{pmatrix}
    \]
    \end{lstlisting}
    \[
    \begin{pmatrix}
    a & b \\
    c & d
    \end{pmatrix}
    \]    

    \item Matrizes com colchetes (usando \verb|bmatrix|):
    \begin{lstlisting}[language=tex, caption=Matriz com colchetes]
    \[
    \begin{bmatrix}
    X & Y \\
    Z & W
    \end{bmatrix}
    \]
    \end{lstlisting}
    \[
    \begin{bmatrix}
    X & Y \\
    Z & W
    \end{bmatrix}
    \] 
\end{itemize}

\section{Ambientes avançados}

\begin{itemize}
    \item Equações numeradas:
    \begin{lstlisting}[language=tex, caption=Equação numerada]
    \begin{equation}
        \label{eq:energia}
        E = mc^{2}
    \end{equation}
    \end{lstlisting}
    \begin{equation}
        \label{eq:energia}
        E = mc^{2}
    \end{equation}

    \item Alinhamento de equações (ambiente \verb|align|):
    \begin{lstlisting}[language=tex, caption=Equações alinhadas]
    \begin{align}
        x & + y = 5 \\
        2x - y = 3 &
    \end{align}
    \end{lstlisting}
    \begin{align}
        x & + y = 5 \\
        2x - y = 3 &
    \end{align}
\end{itemize}

Para não enumerar as equações, use o ambiente \verb|align*|:

\begin{lstlisting}[language=tex, caption=Equações alinhadas e não enumeradas]
\begin{align*}
    x & + y = 5 \\
    2x - y = 3 &
\end{align*}
\end{lstlisting}
\begin{align*}
    x & + y = 5 \\
    2x - y = 3 &
\end{align*}

\section{Sistemas de equações}

Use o ambiente \verb|cases| (requer o pacote \verb|amsmath|):

\begin{lstlisting}[language=tex, caption=Sistema de equações]
    \[
    f(x) = 
    \begin{cases}
    x^{2} & \text{se } x \geq 0 \\
    -x & \text{se } x < 0
    \end{cases}
    \]
\end{lstlisting}

\[
f(x) = 
\begin{cases}
x^{2} & \text{se } x \geq 0 \\
-x & \text{se } x < 0
\end{cases}
\]

\section{Dicas práticas}

\begin{itemize}
    \item Pacote essencial:
    \begin{lstlisting}[language=tex, caption=Carregue o pacote \texttt{amsmath}]
    \usepackage{amsmath} % Carregue sempre!
    \end{lstlisting}
    \item Erros comuns:
    \begin{itemize}
        \item Esquecer de usar \verb|\| antes de símbolos especiais (ex: \verb|\alpha|, não \verb|alpha|).
        \item Não fechar chaves em expoentes/subscritos (ex: \verb|x^{2}| em vez de \verb|x^2|).
    \end{itemize}
\end{itemize}

Dica: Use o \href{https://detexify.kirelabs.org/classify.html}{Detexify} para encontrar símbolos matemáticos desconhecidos!

\bigskip
\noindent\textbf{Transição para o próximo capítulo:}\\[1ex]
Após explorar e praticar as expressões matemáticas, no Capítulo~\ref{cap:6_bibli_cite} você aprenderá a organizar e formatar suas referências bibliográficas, utilizando ferramentas que facilitam a citação e a formatação da bibliografia em LaTeX.


\chapter{Bibliografia e citações em LaTeX}\label{cap:6_bibli_cite}

\chapterquote{``Se pude ver mais longe foi por estar apoiado sobre os ombros de gigantes.''}{Isaac Newton}

A correta citação de referências e a organização da bibliografia são essenciais para a produção de documentos acadêmicos e científicos. O \LaTeX{} oferece ferramentas robustas para gerenciar referências de forma automatizada, garantindo precisão e padronização. O gerenciamento de referências no LaTeX é amplamente facilitado pelo uso de BibTeX e BibLaTeX \cite{overleaf2022guide}.

Estudos recentes destacam o impacto do \LaTeX{} na produção de trabalhos científicos e acadêmicos, demonstrando sua importância na automatização da formatação e na confiabilidade das citações \cite{gilbert2020latex}.

Nos capítulos anteriores, abordamos tabelas (\autoref{cap:3_tabelas}) e, futuramente, gráficos (\autoref{cap:8_graf_diagra}) e apresentações (\autoref{cap:9_beamer(PPX_latec)}), que frequentemente requerem referências bem estruturadas.

% --------

\section{Introdução ao gerenciamento de referências}
\label{sec:cap6_introdução_ao_gerenciamento_de_referências} % Referência cruzada para esta seção

O \LaTeX{} oferece ferramentas robustas para a criação e o gerenciamento automatizado de referências bibliográficas, assegurando consistência e padronização em documentos acadêmicos.

Para realizar uma citação no \LaTeX — seja de uma tabela, imagem ou referência bibliográfica — é necessário, primeiramente, criar um marcador, ou label, que permitirá fazer referência a esse elemento ao longo do texto. No caso de imagens e tabelas, os labels funcionam como apelidos que identificam esses objetos, possibilitando sua citação precisa no conteúdo.

Já para referências bibliográficas (como livros, artigos, sites e revistas), é considerada uma boa prática utilizar um arquivo .bib. Esse arquivo, conforme detalhado no Capítulo \ref{cap:6.6_bib}, armazena os dados das fontes utilizadas e permite que o \LaTeX gere automaticamente tanto as citações ao longo do texto quanto a lista final de referências. Ressalta-se que esse mecanismo é exclusivo para referências textuais, não se aplicando a figuras ou tabelas.

\section{Citações básicas}

Use o comando \verb|\cite{citação}| para citar uma referência.

\begin{lstlisting}[language=tex, caption=Exemplo simples de citação]
    Segundo \cite{einstein}, a teoria da relatividade...
\end{lstlisting}

\section{Criando uma bibliografia manual}

Para documentos pequenos, use o ambiente \verb|thebibliography|:

\begin{lstlisting}[language=tex, caption=Equação inline]
    \begin{thebibliography}{9} % O número 9 indica a largura máxima de rótulos
        \bibitem{einstein}  
            Einstein, A. (1905).  
            \textit{Zur Elektrodynamik bewegter Körper}.  
            Annalen der Physik, 17(10), 891-921.  
    \end{thebibliography}
\end{lstlisting}

\section{Usando o BibTeX (Recomendado)}\label{cap:6.6_bib}

O \textbf{BibTeX} automatiza a criação de bibliografias. Siga os passos:

\begin{enumerate}
    \item Crie um arquivo \verb|.bib| (ex: \verb|referencias.bib|) com entradas no formato:
    \begin{lstlisting}[language=tex, caption=Exemplo de formatação em um arquivo \texttt{.bib}]
    @article{einstein1905,
        author  = "Albert Einstein",
        title   = "Zur Elektrodynamik bewegter Körper",
        journal = "Annalen der Physik",
        year    = "1905",
        volume  = "322",
        pages   = "891--921"
    }
    \end{lstlisting}

    \item Inclua o arquivo no documento:
    \begin{lstlisting}[language=tex, caption=Incluindo o arquivo de bibliografias no documento]
    \bibliographystyle{plain} % Estilo (ex: abnt, ieee, apa)
    \bibliography{referencias} % Nome do arquivo .bib (sem extensão)
    \end{lstlisting}
\end{enumerate}

\section{Estilos de citação}

Altere o estilo com \verb|\bibliographystyle|:

\begin{itemize}
    \item \verb|plain|: Numérico (padrão).
    \item \verb|abnt|: Normas ABNT (requer o pacote \verb|abntex2|).
    \item \verb|apa|: Formato APA.
    \item \verb|ieee|: Padrão IEEE.
\end{itemize}

\begin{lstlisting}[language=tex, caption=Definindo o estilo da bibliografia]
    \bibliographystyle{ieee}  
    \bibliography{referencias}  
\end{lstlisting}

\section{Dicas práticas}
\begin{itemize}
    \item \textbf{Zotero + Better BibTeX}: Exporte referências diretamente para \verb|.bib|.
    \item \textbf{JabRef}: Editor gratuito para gerenciar arquivos \verb|.bib|.
    \item \textbf{DOI para BibTeX}: Sites como \href{https://www.doi2bib.org/}{doi2bib} convertem DOIs em entradas BibTeX.
    \item Erros comuns:
    \begin{itemize}
        \item Referência não aparece: Verifique se a chave em \verb|\cite{citação}| corresponde à entrada no arquivo \verb|.bib|.
        \item Estilo incorreto: Confira se o estilo desejado e adequado está instalado.
    \end{itemize}
\end{itemize}

\bigskip
\noindent\textbf{Transição para o próximo capítulo:}\\[1ex]
Após consolidar o uso de referências e citações, no Capítulo~\ref{cap:7_layout_profissional} você aprenderá a aplicar layouts profissionais que proporcionarão um acabamento refinado e elegante aos seus documentos.
\include{capitulos/7_layout_profissional}
\include{capitulos/8_graf_diagra}  
\chapter{Apresentações com Beamer}\label{cap:9_beamer(PPX_latec)}

\chapterquote{``O design não é apenas o que se vê e se sente, mas como funciona.''}{Steve Jobs}

O LaTeX também pode ser utilizado para criar apresentações dinâmicas e profissionais por meio da classe \texttt{beamer}. Com um sistema modular e personalizável, o Beamer permite a construção de slides que mantêm a consistência visual e a clareza na exposição do conteúdo. Este capítulo explora a estrutura básica de uma apresentação e as principais opções de personalização disponíveis.

Para incorporar tabelas em apresentações, consulte o \autoref{cap:3_tabelas}. Caso precise incluir equações matemáticas, veja o \autoref{cap:5_Exp_matematica}. Além disso, gráficos gerados no LaTeX são abordados no \autoref{cap:8_graf_diagra}.

% --------

\section{Vantagens}
\label{sec:cap9_vantagens} % Referência cruzada para esta seção

\begin{itemize}
    \item Integração perfeita com fórmulas matemáticas.
    \item Templates prontos e altamente customizáveis.
    \item Geração automática de handouts (versão impressa).
\end{itemize}

\begin{lstlisting}[language=tex, caption=Estrutura básica de uma apresentação]
    \documentclass{beamer}
    \usepackage[brazilian]{babel}
    \usetheme{Warsaw} % Tema visual
    \title{Título da Apresentação}
    \author{Seu Nome}
    \date{\today}
    
    \begin{document}
    
    % Slide de título
    \begin{frame}
        \titlepage
    \end{frame}
    
    % Slide de conteúdo
    \begin{frame}{Primeiro Slide}
        \begin{itemize}
            \item Item 1
            \item Item 2
            \pause % Revela os itens sequencialmente
            \item Item 3
        \end{itemize}
    \end{frame}
    
    \end{document}
\end{lstlisting}

\begin{figure}[h!]
  \centering
  \includestandalone[pages=1,width=0.7\linewidth]{exemplos-beamer/estrutura-basica}
  \caption{Estrutura básica de uma apresentação}
\end{figure}

\section{Temas e personalização}

\begin{itemize}
    \item Temas populares:
    \begin{lstlisting}[language=tex, caption=Temas populares]
    \usetheme{Berlin}       % Barras laterais
    \usetheme{Darmstadt}    % Minimalista
    \usetheme{Singapore}    % Cores suaves
    \usetheme{Montpellier}  % Gradientes
    \end{lstlisting}
    \item Cores e fontes:
    \begin{lstlisting}[language=tex, caption=Cores e fontes]
    \usecolortheme{dolphin} % Esquema de cores
    \usefonttheme{serif}     % Fonte serifada
    \end{lstlisting}
\end{itemize}

\section{Seções e sumário}

Adicione estruturas com seções:

\begin{lstlisting}[language=tex, caption=Estrutura seccionada]
    \section{Introdução}
    \begin{frame}{Introdução}
        Conteúdo da introdução...
    \end{frame}
    
    \section{Resultados}
    \begin{frame}{Resultados}
        Gráficos e tabelas...
    \end{frame}
\end{lstlisting}

Slide de sumário com destaque:

\begin{lstlisting}[language=tex, caption=Slide de sumário]
    \begin{frame}{Sumário}
        \tableofcontents[currentsection] % Destaca a seção atual
    \end{frame}
\end{lstlisting}

\section{Blocos e alertas}

Destaque informações com blocos:

\begin{lstlisting}[language=tex, caption=Blocos de destaque]
    \begin{frame}{Blocos}
        \begin{block}{Bloco Padrão}
            Texto informativo.
        \end{block}
        
        \begin{alertblock}{Alerta!}
            Mensagem importante em vermelho.
        \end{alertblock}
        
        \begin{exampleblock}{Exemplo}
            Exemplo prático em verde.
        \end{exampleblock}
    \end{frame}
\end{lstlisting}

\section{Inserindo gráficos e animações}

Animações simples com \verb|\pause|

\begin{lstlisting}[language=tex, caption=Gráfico animado]
    \begin{frame}{Gráfico Animado}
        \begin{figure}
            \includegraphics[width=0.5\textwidth]{grafico1.png}
            \pause
            \caption{Gráfico revelado após um clique.}
        \end{figure}
    \end{frame}
\end{lstlisting}

Overlays complexos com \verb|\uncover|

\begin{lstlisting}[language=tex, caption=Overlays complexos]
    \begin{frame}{Revelação Progressiva}
        \begin{itemize}
            \item<1-> Primeiro ponto (visível desde o início)
            \item<2-> Segundo ponto (visível no segundo clique)
            \item<3-> Terceiro ponto (visível no terceiro clique)
        \end{itemize}
    \end{frame}
\end{lstlisting}

\section{Recursos avançados}

Adicione um logotipo no rodapé:

\begin{lstlisting}[language=tex, caption=Logotipo no rodapé]
    \logo{\includegraphics[height=0.8cm]{logo.png}}
\end{lstlisting}

Footline personalizado:

\begin{lstlisting}[language=tex, caption=Footline customizado]
    \setbeamertemplate{footline}{
        \hfill\insertauthor\quad\insertshorttitle\quad\insertframenumber
    }
\end{lstlisting}

\section{Dicas práticas}

\begin{itemize}
    \item Compile com a opção \verb|handout| para gerar versões impressas:
    \begin{lstlisting}[language=tex, caption=Versão impressa]
    \documentclass[handout]{beamer}
    \usepackage{pgfpages}
    \pgfpagesuselayout{4 on 1}[a4paper,border shrink=5mm]
    \end{lstlisting} 
    \item Pacotes úteis:
    \begin{lstlisting}[language=tex, caption=Pacotes úteis]
    \usepackage{hyperref} % Links clicáveis
    \usepackage{tcolorbox} % Caixas estilizadas
    \end{lstlisting} 
    \item Recursos adicionais:
    \begin{itemize}
        \item Templates do overleaf: \href{https://www.overleaf.com/latex/templates/tagged/presentation}{modelos prontos para apresentações}.
        \item \href{https://linorg.usp.br/CTAN/macros/latex/contrib/beamer/doc/beameruserguide.pdf}{Documentação oficial - manual do Beamer}.
    \end{itemize}
    \item Erros comuns:
    \begin{itemize}
        \item Esquecer de fechar ambientes (\verb|frame|, \verb|block|).
        \item Usar \verb|\pause| excessivamente (pode distrair o público).
    \end{itemize}
    
\end{itemize}

\bigskip
\noindent\textbf{Considerações Finais:}\\[1ex]
Agora que você compreendeu a criação de apresentações profissionais com o Beamer, esperamos que este livro tenha contribuído para ampliar seus conhecimentos em \LaTeX{}. Continue praticando e explorando as inúmeras possibilidades que esta ferramenta oferece para transformar suas ideias em documentos e apresentações impactantes!

\bibliographystyle{alpha} % Estilo 
\bibliography{referencias} 
    
\newpage
\appendix
\label{sec:apendice}

\chapter{Símbolos matemáticos}

\begin{longtable}{|c|L{2.6cm}|L{6.5cm}|}
\caption{Símbolos matemáticos comuns}
\label{tab:simbolos} \\
\hline
\textbf{Símbolo} & \textbf{Código LaTeX} & \textbf{Descrição} \\
\hline
\endfirsthead
\hline
\textbf{Símbolo} & \textbf{Código LaTeX} & \textbf{Descrição} \\
\hline
\endhead

% Operadores Matemáticos
$\cdot$ & \verb|\cdot| & Produto escalar \\
$\tfrac{a}{b}$ & \verb|\tfrac{a}{b}| & Divisão \\
$\neq$ & \verb|\neq| & Diferente \\
$\leq$ & \verb|\leq| & Menor ou igual \\
$\geq$ & \verb|\geq| & Maior ou igual \\
$\approx$ & \verb|\approx| & Aproximadamente igual \\
$\propto$ & \verb|\propto| & Proporcional \\
$\infty$ & \verb|\infty| & Infinito \\

% Álgebra e Conjuntos
$\in$ & \verb|\in| & Pertence ao conjunto \\
$\notin$ & \verb|\notin| & Não pertence \\
$\subset$ & \verb|\subset| & Subconjunto \\
$\cup$ & \verb|\cup| & União \\
$\cap$ & \verb|\cap| & Interseção \\
$\forall$ & \verb|\forall| & Para todo \\
$\exists$ & \verb|\exists| & Existe \\

% Funções e Operações
$\sqrt{x}$ & \verb|\sqrt{x}| & Raiz quadrada \\
$\int$ & \verb|\int| & Integral \\
$\sum$ & \verb|\sum| & Somatório \\
$\lim$ & \verb|\lim| & Limite \\
$\log$ & \verb|\log| & Logaritmo \\
$\ln$ & \verb|\ln| & Logaritmo natural \\
$\exp$ & \verb|\exp| & Exponencial \\

% Vetores e Matrizes
$\vec{v}$ & \verb|\vec{v}| & Vetor \\
$\mathbf{A}$ & \verb|\mathbf{A}| & Matriz (ou vetor em negrito) \\
$\nabla$ & \verb|\nabla| & Operador nabla (gradiente) \\
$\cdot$ & \verb|\cdot| & Produto escalar \\
$\times$ & \verb|\times| & Produto vetorial \\

% Símbolos Físicos e de Engenharia
$\Delta$ & \verb|\Delta| & Variação \\
$\omega$ & \verb|\omega| & Velocidade angular \\
$\mu$ & \verb|\mu| & Coeficiente de atrito / Permeabilidade \\
$\sigma$ & \verb|\sigma| & Tensão (stress) ou condutividade \\
$\varepsilon$ & \verb|\varepsilon| & Deformação (strain) ou permissividade \\
$\tau$ & \verb|\tau| & Torque ou tempo de relaxamento \\

% Constantes
$\pi$ & \verb|\pi| & Constante pi \\

% Unidades e Outros
$\degree$ & \verb|\degree| (com pacote \texttt{gensymb}) & Graus \\
$\Omega$ & \verb|\Omega| & Ohm (resistência) \\
$\rightarrow$ & \verb|\rightarrow| & Tendendo / Implicação \\
$\leftrightarrow$ & \verb|\leftrightarrow| & Equivalência \\
\hline
\end{longtable}

\end{document}