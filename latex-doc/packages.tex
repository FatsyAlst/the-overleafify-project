% NOME DO PDF %
% \usepackage[
%   pdftitle={The Overleafify Project}
% ]{hyperref}


% COMANDOS PARA \LaTeX\ E \TeX\ COM ESPAÇAMENTO AUTOMÁTICO
%-------------------------------------------------------
\usepackage{xspace}  % Carrega o pacote xspace, que insere automaticamente um espaço após macros
\newcommand{\LaTeXcmd}{\LaTeX\xspace} % Define \LaTeXcmd como um atalho para \LaTeX, e usa \xspace para adicionar um espaço quando o próximo caractere for alfanumérico, evitando precisar escrever {} ou \, toda vez.
\newcommand{\TeXcmd}{\TeX\xspace} % Define \TeXcmd análogo a \LaTeXcmd, mas para o logo \TeX.

% FORMATO DE PÁGINA E LÍNGUA
%--------------------------------------------------------
\usepackage[a5paper,top=1.5cm,bottom=2cm,left=2cm,right=2cm,marginparwidth=1.75cm]{geometry} 
% Define margens e tamanho da página em formato A5

\usepackage[utf8]{inputenc}   % Suporte a caracteres especiais (acentuação, ç, etc.) no código fonte
\usepackage[T1]{fontenc}      % Melhora o suporte de hifenização e acentuação na saída do PDF
\usepackage[brazilian]{babel} % Adapta o documento para o português (traduções de nomes e hifenização)

% PACOTES MATEMÁTICOS E FONTES
%--------------------------------------------------------
\usepackage{amsmath, amsthm, amssymb} % Pacotes para equações, símbolos matemáticos e ambientes teorema
\usepackage{fouriernc}               % Usa a fonte "Fourier New Century" com bom suporte para matemática

% IMAGENS E GRÁFICOS
%--------------------------------------------------------
\usepackage{graphicx} % Permite inserir imagens
    \graphicspath{{imagens/}} % Define o caminho padrão das imagens

% CORES PERSONALIZADAS
%--------------------------------------------------------
\usepackage[dvipsnames]{xcolor} % Fornece uma paleta de cores estendida
    \definecolor{mygreen}{rgb}{0,0.6,0}    % Verde personalizado
    \definecolor{mygray}{rgb}{0.5,0.5,0.5} % Cinza personalizado
    \definecolor{mymauve}{rgb}{0.58,0,0.82} % Roxo (malva) personalizado

% GERAÇÃO DE TEXTO FICTÍCIO
%--------------------------------------------------------
\usepackage{lipsum} % Gera texto fictício para testes (ex: \lipsum[1-2])

% COMPILAÇÃO EXTERNA DOS CÓDIGOS BEAMER
%--------------------------------------------------------
\usepackage[mode=buildnew]{standalone} % compila se o PDF estiver desatualizado
% ou ainda: \usepackage[mode=buildmissing]{standalone}

% SUMÁRIO PERSONALIZADO
%--------------------------------------------------------
\usepackage{tocloft} % Permite personalizar o sumário (estilo, títulos, etc.)

% TIKZ E PACOTES PARA DIAGRAMAS E GRÁFICOS
%--------------------------------------------------------
\usepackage{tikz}           % Pacote principal para gráficos vetoriais
\usepackage{pgfplots}       % Gráficos de funções e dados
\usepackage{pgfplotstable}  % Leitura e plotagem de tabelas de dados
    \usetikzlibrary{shapes, arrows.meta, positioning, backgrounds} % Bibliotecas do TikZ para formas, setas e posicionamento
    \pgfplotsset{compat=newest} % Usa a versão mais recente do PGFPlots
    \usetikzlibrary{shapes.geometric, calc, arrows, shapes, trees} % Mais formas, cálculos e árvores
    \usetikzlibrary{arrows.meta} % Estilo de setas mais moderno

\usepackage{tikz-3dplot} % Permite desenhar em 3D com o TikZ

% CAIXAS COLORIDAS PERSONALIZADAS (EX: para exercícios)
%--------------------------------------------------------
\usepackage[most]{tcolorbox} % Criação de caixas coloridas avançadas
\tcbuselibrary{skins,raster} % Adiciona efeitos visuais e layouts em grade para tcolorbox

% FÍSICA E FERRAMENTAS MATEMÁTICAS ADICIONAIS
%--------------------------------------------------------
\usepackage{physics} % Facilita a digitação de expressões físicas (ex: \dv, \qty, etc.)

% ELEMENTOS NO FUNDO DA PÁGINA (EX: CAPA COM IMAGEM)
%--------------------------------------------------------
\usepackage{eso-pic} % Permite adicionar elementos gráficos ao fundo da página (ex: \AddToShipoutPicture)

% ALINHAMENTO DE TEXTO
%--------------------------------------------------------
\usepackage{ragged2e} % Permite usar comandos como \justifying, \RaggedRight

% MÚLTIPLAS COLUNAS
%--------------------------------------------------------
\usepackage{multicol} % Permite dividir texto em várias colunas

% LISTAS PERSONALIZADAS
%--------------------------------------------------------
\usepackage[shortlabels]{enumitem} % Criação de listas com rótulos personalizados (como [a], [i], etc.)

% AMBIENTE PARA CÓDIGO-FONTE
%--------------------------------------------------------
\usepackage{listings} % Permite incluir e formatar código-fonte
    \lstset{
        language=tex,               % Linguagem usada no código (aqui: LaTeX)
        basicstyle=\ttfamily\small, % Fonte monoespaçada pequena
        keywordstyle=\color{blue},  % Cor das palavras-chave
        commentstyle=\color{gray},  % Cor dos comentários
        stringstyle=\color{red},    % Cor das strings
        frame=single,               % Moldura em volta do código
        breaklines=true,            % Quebra linhas automaticamente
        captionpos=b                % Posição da legenda (b = abaixo)
    }

\renewcommand{\lstlistingname}{Código} % Muda o nome padrão de "Listing" para "Código"
    % Suporte para acentuação correta dentro do ambiente de código
    \lstset{
        literate={á}{{\'a}}1 {é}{{\'e}}1 {í}{{\'i}}1 {ó}{{\'o}}1 {ú}{{\'u}}1
                 {ã}{{\~a}}1 {õ}{{\~o}}1 {â}{{\^a}}1 {ê}{{\^e}}1 {ô}{{\^o}}1
                 {À}{{\`A}}1 {à}{{\`a}}1 {ü}{{\"u}}1 {ç}{{\c{c}}}1 {Ç}{{\c{C}}}1
                 {ö}{{\"o}}1
    }

% ESPAÇAMENTO ENTRE LINHAS
%--------------------------------------------------------
\usepackage{setspace} % Permite usar \singlespacing, \onehalfspacing, \doublespacing

% TABELAS AVANÇADAS
%--------------------------------------------------------
\usepackage{tabularx}   % Tabelas com largura ajustável
\usepackage{multirow}   % Mescla de células em linhas diferentes
\usepackage{booktabs}   % Regras horizontais com aparência profissional
\usepackage{subcaption} % Permite usar subfiguras dentro de uma figura

% HIPERLINKS E NAVEGAÇÃO
%--------------------------------------------------------
\usepackage{hyperref} % Permite a criação de links clicáveis
    \definecolor{mylinkcolor}{HTML}{2D4802} % Cor dos links internos (sumário, referências)
    \definecolor{mycitecolor}{HTML}{8EB12D} % Cor dos links de citação
    \definecolor{myurlcolor}{HTML}{638204}  % Cor dos links externos (URLs)

    \hypersetup{
      linkcolor  = mylinkcolor,
      citecolor  = mycitecolor,
      urlcolor   = myurlcolor,
      colorlinks = true       % Ativa links coloridos (sem molduras)
    }
% REFERÊNCIAS INTELIGENTES
%--------------------------------------------------------
\usepackage{cleveref} % Melhora as referências cruzadas (ex: "ver Capítulo 3" automaticamente)

% CRIAÇÃO DE ARQUIVOS EMBUTIDOS
%--------------------------------------------------------
\usepackage{filecontents} % Permite criar arquivos diretamente no .tex (útil para bib ou dados)

% ÍCONES (EX: redes sociais, símbolos visuais)
%--------------------------------------------------------
\usepackage{fontawesome5} % Permite usar ícones como \faGithub, \faEnvelope, etc.

% EFEITO DE SOMBRA EM OBJETOS TIKZ
\usetikzlibrary{shadows.blur} % Permite adicionar sombra borrada nos gráficos TikZ

% CAIXAS COM LARGURA VARIÁVEL
%--------------------------------------------------------
\usepackage{varwidth} % Permite criar caixas que ajustam sua largura ao conteúdo

% Tabelas personalizadas
%------------------------------------------------------
\usepackage{longtable}
\usepackage{amsmath, amssymb}
\usepackage{geometry}
\usepackage{gensymb} %usei para adicionar o simbolo de graus na tabela
\usepackage{array}
\usepackage{lscape} % Opcional: útil se quiser girar a tabela



% Define um novo tipo de coluna para texto com quebra automática
\newcolumntype{L}[1]{>{\raggedright\arraybackslash}p{#1}}