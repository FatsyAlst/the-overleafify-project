\chapter{Criação de tabelas}
\label{cap:3_tabelas}

\chapterquote{``Acima de tudo, mostre os dados.''}{Edward R. Tufte}

Tabelas são elementos essenciais para a organização e apresentação de dados em documentos técnicos e científicos. O \LaTeX{} oferece um sistema flexível para criar tabelas com controle preciso sobre alinhamento, bordas, espaçamento e personalização. Este capítulo aborda a sintaxe do ambiente \texttt{tabular}, explicando como estruturar e formatar tabelas de maneira eficiente.

% --------

\section{Tabelas básicas com \texttt{tabular}}
\label{sec:cap3_tabelas_básicas_com_tabular} % Referência cruzada para esta seção

O ambiente \verb|tabular| é a base para criar tabelas em \LaTeX{}, oferecendo controle sobre alinhamento e espaçamento \cite{mittelbach2004latex}.

\begin{lstlisting}[language=tex, caption=Sintaxe básica para criação de tabelas]
    \begin{tabular}{colunas}
        Conteúdo das células & separado por & \\
        Linhas separadas por \\ 
    \end{tabular}
\end{lstlisting}

\newpage

\begin{lstlisting}[language=tex, caption=Exemplo básico de tabela]
    \begin{center}
        \begin{tabular}{|c|c|c|}  % Colunas centralizadas (c) com bordas (|)
            \hline
            Nome & Idade & Cidade \\ 
            \hline
            João & 25 & São Paulo \\ 
            Maria & 30 & Rio de Janeiro \\ 
            \hline
        \end{tabular}
    \end{center}
\end{lstlisting}

\begin{center}
    \begin{tabular}{|c|c|c|}  % Colunas centralizadas (c) com bordas (|)
        \hline
        Nome & Idade & Cidade \\ 
        \hline
        João & 25 & São Paulo \\ 
        Maria & 30 & Rio de Janeiro \\ 
        \hline
    \end{tabular}
\end{center}

\section{Tipos de colunas}

\begin{itemize}
    \item \verb|c|: Centralizada.
    \item \verb|l|: Alinhada à esquerda.
    \item \verb|r|: Alinhada à direita;
    \item \verb|||: Adiciona bordas verticais
    \item \verb|p{largura}|: Coluna com largura fixa
\end{itemize}

\begin{lstlisting}[language=tex, caption=Exemplo com alinhamento]
    \begin{tabular}{l r p{4cm}}
        Nome (esquerda) & Preço (direita) & Descrição (largura fixa) \\ 
        \hline
        Livro & R\$ 50,00 & Um livro sobre LaTeX \\ 
        Caneta & R\$ 2,50 & Caneta azul \\ 
    \end{tabular}
\end{lstlisting}

\begin{tabular}{l r p{4cm}}
    Nome (esquerda) & Preço (direita) & Descrição (largura fixa) \\ 
    \hline
    Livro & R\$ 50,00 & Um livro sobre \LaTeX{} \\ 
    Caneta & R\$ 2,50 & Caneta azul \\ 
\end{tabular}

\section{Pacote \texttt{tabularx} para largura ajustável}

Permite criar tabelas que se ajustam à largura do texto.

\begin{lstlisting}[language=tex, caption=Uso do pacote \texttt{tabularx}]
\begin{tabularx}{\textwidth}{|X|X|X|} % Colunas X dividem o espaço igualmente
    \hline
    Cabeçalho 1 & Cabeçalho 2 & Cabeçalho 3 \\ 
    \hline
    Texto longo que se ajusta automaticamente & Dado 2 & Dado 3 \\ 
    \hline
\end{tabularx}
\end{lstlisting}


\begin{tabularx}{\textwidth}{|X|X|X|} % Colunas X dividem o espaço igualmente
    \hline
    Cabeçalho 1 & Cabeçalho 2 & Cabeçalho 3 \\ 
    \hline
    Texto longo que se ajusta automaticamente & Dado 2 & Dado 3 \\ 
    \hline
\end{tabularx}


\section{Linhas horizontais e verticais}

\begin{itemize}
    \item \verb|\hline|: Linha horizontal.
    \item \verb|cline{i-j}|: Linha horizontal parcial.
    \item \verb|||: Bordas verticais (ex: \texttt{|c|c|c|}).
\end{itemize}

\begin{lstlisting}[language=tex, caption=Exemplo de criação de tabela]
    \begin{center}
        \begin{tabular}{|l|c|r|}
            \hline
            \multicolumn{3}{|c|}{Título da Tabela} \\ 
            \hline
            Item & Quantidade & Preço \\ 
            \cline{1-2} % Linha parcial
            Livro & 2 & R\$ 100,00 \\ 
            \hline
        \end{tabular}
    \end{center}
\end{lstlisting}

\begin{center}
    \begin{tabular}{|l|c|r|}
        \hline
        \multicolumn{3}{|c|}{Título da Tabela} \\ 
        \hline
        Item & Quantidade & Preço \\ 
        \cline{1-2} % Linha parcial
        Livro & 2 & R\$ 100,00 \\ 
        \hline
    \end{tabular}
\end{center}

\section{Mesclando células}

\begin{itemize}
    \item \verb|\multicolumn{n}{alinhamento}{texto}|: Mescla colunas.
    \item \verb|\multirow{n}{largura}{texto}|: Mescla linhas (requer o pacote \verb|multirow|).
\end{itemize}

\begin{lstlisting}[language=tex, caption=Exemplo de mesclagem]
\begin{center}
    \begin{tabular}{|c|c|c|}
        \hline
        \multirow{2}{*}{Categoria} & \multicolumn{2}{c|}{Dados} \\ 
        \cline{2-3}
        & Produto & Preço \\ 
        \hline
        Livros & LaTeX Guide & R\$ 80,00 \\ 
        \hline
    \end{tabular}
\end{center}
\end{lstlisting}

\begin{center}
    \begin{tabular}{|c|c|c|}
        \hline
        \multirow{2}{*}{Categoria} & \multicolumn{2}{c|}{Dados} \\ 
        \cline{2-3}
        & Produto & Preço \\ 
        \hline
        Livros & LaTeX Guide & R\$ 80,00 \\ 
        \hline
    \end{tabular}
\end{center}

\section{Tabelas profissionais com \texttt{booktabs}}

O pacote \verb|booktabs| remove bordas excessivas e melhora a estética:

\begin{lstlisting}[language=tex, caption=Exemplo de mesclagem]
\begin{center}
    \begin{tabular}{lcc}
        \toprule
        Nome & Nota 1 & Nota 2 \\ 
        \midrule
        João & 8,5 & 9,0 \\ 
        Maria & 7,0 & 8,5 \\ 
        \bottomrule
    \end{tabular}
\end{center}
\end{lstlisting}

\begin{center}
    \begin{tabular}{lcc}
        \toprule
        Nome & Nota 1 & Nota 2 \\ 
        \midrule
        João & 8,5 & 9,0 \\ 
        Maria & 7,0 & 8,5 \\ 
        \bottomrule
    \end{tabular}
\end{center}

Dica: Evite usar \verb|\hline| com \verb|booktabs| para manter o estilo limpo.

\section{Dicas práticas}

\begin{itemize}
    \item Pacotes úteis:
    \begin{lstlisting}[language=tex, caption=Pacotes úteis e interessantes para usar com tabelas]
    \usepackage{array}       % Mais controle sobre colunas
    \usepackage{graphicx}    % Inserir imagens em células
    \end{lstlisting}
    \item Erros comuns:
    \begin{itemize}
        \item Esquecer de adicionar \& entre células.
        \item Usar \verb|\\| no final da última linha (desnecessário).
    \end{itemize}
\end{itemize}

\bigskip
\noindent\textbf{Transição para o próximo capítulo:}\\[1ex]
Agora que você aprendeu a organizar e formatar tabelas de forma clara e elegante, no Capítulo~\ref{cap:4_manip_img} vamos mergulhar na Manipulação de Imagens em \LaTeX{}. Você descobrirá como inserir, ajustar e posicionar imagens de modo que elas complementem e enriqueçam seus documentos.
    
