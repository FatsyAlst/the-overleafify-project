\chapter{Apresentações com Beamer}\label{cap:9_beamer(PPX_latec)}

\chapterquote{``O design não é apenas o que se vê e se sente, mas como funciona.''}{Steve Jobs}

O LaTeX também pode ser utilizado para criar apresentações dinâmicas e profissionais por meio da classe \texttt{beamer}. Com um sistema modular e personalizável, o Beamer permite a construção de slides que mantêm a consistência visual e a clareza na exposição do conteúdo. Este capítulo explora a estrutura básica de uma apresentação e as principais opções de personalização disponíveis.

Para incorporar tabelas em apresentações, consulte o \autoref{cap:3_tabelas}. Caso precise incluir equações matemáticas, veja o \autoref{cap:5_Exp_matematica}. Além disso, gráficos gerados no LaTeX são abordados no \autoref{cap:8_graf_diagra}.

% --------

\section{Vantagens}
\label{sec:cap9_vantagens} % Referência cruzada para esta seção

\begin{itemize}
    \item Integração perfeita com fórmulas matemáticas.
    \item Templates prontos e altamente customizáveis.
    \item Geração automática de handouts (versão impressa).
\end{itemize}

\begin{lstlisting}[language=tex, caption=Estrutura básica de uma apresentação]
    \documentclass{beamer}
    \usepackage[brazilian]{babel}
    \usetheme{Warsaw} % Tema visual
    \title{Título da Apresentação}
    \author{Seu Nome}
    \date{\today}
    
    \begin{document}
    
    % Slide de título
    \begin{frame}
        \titlepage
    \end{frame}
    
    % Slide de conteúdo
    \begin{frame}{Primeiro Slide}
        \begin{itemize}
            \item Item 1
            \item Item 2
            \pause % Revela os itens sequencialmente
            \item Item 3
        \end{itemize}
    \end{frame}
    
    \end{document}
\end{lstlisting}

\begin{figure}[h!]
  \centering
  \includestandalone[pages=1,width=0.7\linewidth]{exemplos-beamer/estrutura-basica}
  \caption{Estrutura básica de uma apresentação}
\end{figure}

\section{Temas e personalização}

\begin{itemize}
    \item Temas populares:
    \begin{lstlisting}[language=tex, caption=Temas populares]
    \usetheme{Berlin}       % Barras laterais
    \usetheme{Darmstadt}    % Minimalista
    \usetheme{Singapore}    % Cores suaves
    \usetheme{Montpellier}  % Gradientes
    \end{lstlisting}
    \item Cores e fontes:
    \begin{lstlisting}[language=tex, caption=Cores e fontes]
    \usecolortheme{dolphin} % Esquema de cores
    \usefonttheme{serif}     % Fonte serifada
    \end{lstlisting}
\end{itemize}

\section{Seções e sumário}

Adicione estruturas com seções:

\begin{lstlisting}[language=tex, caption=Estrutura seccionada]
    \section{Introdução}
    \begin{frame}{Introdução}
        Conteúdo da introdução...
    \end{frame}
    
    \section{Resultados}
    \begin{frame}{Resultados}
        Gráficos e tabelas...
    \end{frame}
\end{lstlisting}

Slide de sumário com destaque:

\begin{lstlisting}[language=tex, caption=Slide de sumário]
    \begin{frame}{Sumário}
        \tableofcontents[currentsection] % Destaca a seção atual
    \end{frame}
\end{lstlisting}

\section{Blocos e alertas}

Destaque informações com blocos:

\begin{lstlisting}[language=tex, caption=Blocos de destaque]
    \begin{frame}{Blocos}
        \begin{block}{Bloco Padrão}
            Texto informativo.
        \end{block}
        
        \begin{alertblock}{Alerta!}
            Mensagem importante em vermelho.
        \end{alertblock}
        
        \begin{exampleblock}{Exemplo}
            Exemplo prático em verde.
        \end{exampleblock}
    \end{frame}
\end{lstlisting}

\section{Inserindo gráficos e animações}

Animações simples com \verb|\pause|

\begin{lstlisting}[language=tex, caption=Gráfico animado]
    \begin{frame}{Gráfico Animado}
        \begin{figure}
            \includegraphics[width=0.5\textwidth]{grafico1.png}
            \pause
            \caption{Gráfico revelado após um clique.}
        \end{figure}
    \end{frame}
\end{lstlisting}

Overlays complexos com \verb|\uncover|

\begin{lstlisting}[language=tex, caption=Overlays complexos]
    \begin{frame}{Revelação Progressiva}
        \begin{itemize}
            \item<1-> Primeiro ponto (visível desde o início)
            \item<2-> Segundo ponto (visível no segundo clique)
            \item<3-> Terceiro ponto (visível no terceiro clique)
        \end{itemize}
    \end{frame}
\end{lstlisting}

\section{Recursos avançados}

Adicione um logotipo no rodapé:

\begin{lstlisting}[language=tex, caption=Logotipo no rodapé]
    \logo{\includegraphics[height=0.8cm]{logo.png}}
\end{lstlisting}

Footline personalizado:

\begin{lstlisting}[language=tex, caption=Footline customizado]
    \setbeamertemplate{footline}{
        \hfill\insertauthor\quad\insertshorttitle\quad\insertframenumber
    }
\end{lstlisting}

\section{Dicas práticas}

\begin{itemize}
    \item Compile com a opção \verb|handout| para gerar versões impressas:
    \begin{lstlisting}[language=tex, caption=Versão impressa]
    \documentclass[handout]{beamer}
    \usepackage{pgfpages}
    \pgfpagesuselayout{4 on 1}[a4paper,border shrink=5mm]
    \end{lstlisting} 
    \item Pacotes úteis:
    \begin{lstlisting}[language=tex, caption=Pacotes úteis]
    \usepackage{hyperref} % Links clicáveis
    \usepackage{tcolorbox} % Caixas estilizadas
    \end{lstlisting} 
    \item Recursos adicionais:
    \begin{itemize}
        \item Templates do overleaf: \href{https://www.overleaf.com/latex/templates/tagged/presentation}{modelos prontos para apresentações}.
        \item \href{https://linorg.usp.br/CTAN/macros/latex/contrib/beamer/doc/beameruserguide.pdf}{Documentação oficial - manual do Beamer}.
    \end{itemize}
    \item Erros comuns:
    \begin{itemize}
        \item Esquecer de fechar ambientes (\verb|frame|, \verb|block|).
        \item Usar \verb|\pause| excessivamente (pode distrair o público).
    \end{itemize}
    
\end{itemize}

\bigskip
\noindent\textbf{Considerações Finais:}\\[1ex]
Agora que você compreendeu a criação de apresentações profissionais com o Beamer, esperamos que este livro tenha contribuído para ampliar seus conhecimentos em \LaTeX{}. Continue praticando e explorando as inúmeras possibilidades que esta ferramenta oferece para transformar suas ideias em documentos e apresentações impactantes!