\section{Tipos de documentos}

\begin{frame}[fragile]{Estrutura básica de um documento}
\begin{lstlisting}[caption=Exemplo de estrutura mínima]
    \documentclass{article} % Pré-âmbulo
    \begin{document}         % Início do corpo
        Olá, mundo!          % Conteúdo
    \end{document}           % Fim do corpo
    \begin{appendices}       % Pós-âmbulo
\end{lstlisting}
\end{frame}

\begin{frame}{Classes de documentos}
  O \LaTeX{} oferece diferentes classes de documentos para atender a diversos propósitos. As principais são:
  \begin{enumerate}
    \item \texttt{article}: Ideal para artigos científicos, relatórios curtos e documentos simples.
    \item \texttt{report}: Usado para relatórios técnicos, monografias ou documentos com capítulos.
    \item \texttt{book}: Projetado para livros, teses e dissertações, com suporte a capítulos, partes e elementos pré-textuais.
    \item \texttt{beamer}: Para criar apresentações de slides.
  \end{enumerate}
\end{frame}

\begin{frame}{Classes de documentos}
  O \LaTeX{} oferece diferentes classes de documentos para atender a diversos propósitos. As principais são:
  \begin{enumerate}
    \item \texttt{article}: Ideal para artigos científicos, relatórios curtos e documentos simples.
    \item \texttt{report}: Usado para relatórios técnicos, monografias ou documentos com capítulos.
    \item \texttt{book}: Projetado para livros, teses e dissertações, com suporte a capítulos, partes e elementos pré-textuais.
    \item \texttt{beamer}: Para criar apresentações de slides.
      \begin{itemize}
        \item Sim, essa apresentação foi feita usando a classe Beamer.
      \end{itemize}
  \end{enumerate}
\end{frame}
