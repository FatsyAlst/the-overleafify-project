\section{Exercícios}

\begin{frame}{Exercício 1}
  \begin{block}{Criação básica de documento}
    \begin{enumerate}
      \item Crie um novo documento \LaTeX{} utilizando uma das classes discutidas (\texttt{article}, \texttt{report}, \texttt{book}).
      \item No pré-âmbulo, personalize a diagramação e fonte:
        \begin{itemize}
          \item Utilize o pacote \texttt{geometry} para configurar as margens.
          \item Escolha um pacote de fontes (\texttt{palatino}, \texttt{kpfonts}, \texttt{fouriernc}).
        \end{itemize}
      \item No corpo do documento, insira conteúdo para visualizar as personalizações.
    \end{enumerate}
  \end{block}
\end{frame}

\begin{frame}{Exercício 2}
  \begin{block}{Formatação básica de texto}
    \begin{enumerate}
      \item Crie um novo documento \LaTeX{} ao seu gosto.
      \item No corpo do documento, experimente os seguintes comandos de formatação:
        \begin{itemize}
          \item Negrito utilizando \texttt{\textbackslash textbf\{\}}.
          \item Itálico utilizando \texttt{\textbackslash textit\{\}}.
          \item Sublinhado utilizando \texttt{\textbackslash underline\{\}}
        \end{itemize}
      \item Use diferentes tamanhos de fontes com os comandos:
        \begin{itemize}
          \item \texttt{\textbackslash tiny}, \texttt{\textbackslash scriptsize}, \texttt{\textbackslash footnotesize}.
          \item \texttt{\textbackslash small}, \texttt{\textbackslash normalsize}, \texttt{\textbackslash large}.
          \item \texttt{\textbackslash Large}, \texttt{\textbackslash LARGE}, \texttt{\textbackslash huge}, \texttt{\textbackslash Huge}.
        \end{itemize}
      \item Experimente mudar a cor do texto usando o pacote \texttt{xcolor} e o comando \texttt{\textbackslash textcolor\{cor\}\{texto\}}.
    \end{enumerate}
  \end{block}
\end{frame}

\begin{frame}{Exerício 3}
  \begin{block}{Criação de tabela simples}
    \begin{enumerate}
      \item Crie um novo documento \LaTeX{}.
      \item No corpo do documento, crie uma tabela simples utilizando o ambiente \texttt{tabular}:
        \begin{itemize}
          \item Defina três colunas e insira cabeçalhos e dados.
          \item Utilize bordas nas células da tabela.
        \end{itemize}
      \item Compile o documento e verifique o resultado.
    \end{enumerate}
  \end{block}
\end{frame}

\begin{frame}{Exercício 4}
  \begin{block}{Símbolos e expressões matemáticas}
    \begin{enumerate}
      \item Crie um novo documento \LaTeX{}.
      \item No corpo do documento, insira expressões matemáticas com:
        \begin{itemize}
          \item Símbolos básicos (letras gregas e operadores).
          \item Operadores binários e relacionais.
          \item Frações, expoentes e índices.
        \end{itemize}
    \end{enumerate}
  \end{block}
\end{frame}

\begin{frame}{Exercício 5}
  \begin{block}{Inserção de imagens}
    \begin{itemize}
      \item Crie um novo documento \LaTeX{}.
      \item No pré-âmbulo, inclua o pacote \texttt{graphicx} com o comando \texttt{\textbackslash usepackage\{graphicx\}}.
      \item No corpo do documento, insira uma imagem utilizando o comando \texttt{\textbackslash includegraphics\{caminho/para/imagem\}}.
      \item Utilize imagens nos formatos suportados (\texttt{.jpeg}, \texttt{.png}, \texttt{.pdf}).
    \end{itemize}
  \end{block}
\end{frame}
