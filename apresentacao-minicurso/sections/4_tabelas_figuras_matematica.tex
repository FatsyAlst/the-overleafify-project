\section{Tabelas, figuras e expressões matemáticas}

% -----------
% Tabelas – conteúdo original abaixo
% -----------

\begin{frame}[fragile]{Tabelas básicas com \texttt{tabular}}
\begin{lstlisting}[language=tex]
\begin{tabular}{|c|c|c|}
    \hline
    Nome & Idade & Cidade \\
    \hline
    João & 25 & SP \\
    Maria & 30 & RJ \\
    \hline
\end{tabular}
\end{lstlisting}

  \begin{block}{Exemplo de tabela com \texttt{tabular}}
    \centering
    \begin{tabular}{|c|c|c|}
      \hline
      Nome & Idade & Cidade \\
      \hline
      João & 25 & SP \\
      Maria & 30 & RJ \\
      \hline
    \end{tabular}
  \end{block}

\end{frame}

\begin{frame}{Tipos de colunas}
  \begin{block}{Argumentos para tipos de colunas}
    \begin{itemize}
      \item \texttt{l}: alinhado à esquerda
      \item \texttt{c}: centralizado
      \item \texttt{r}: alinhado à direita
      \item \texttt{p\{largura\}}: largura fixa
      \item \texttt{|}: bordas verticais
    \end{itemize}
  \end{block}

  \begin{block}{Exemplo com alinhamento}
    \centering
    \begin{tabular}{l r p{4cm}}
      Nome (esquerda) & Preço (direita) & Descrição (largura fixa) \\
      \hline
      Livro & R\$ 50,00 & Um livro sobre \LaTeX{} \\
      Caneta & R\$ 2,50 & Caneta azul \\
    \end{tabular}
  \end{block}
\end{frame}

\begin{frame}[fragile]{Largura ajustável: \texttt{tabularx}}
\begin{lstlisting}[language=tex]
\usepackage{tabularx}
\centering
\begin{tabularx}{\textwidth}{|X|c|r|} % Colunas X dividem o espaço igualmente
    \hline
    Cabeçalho 1 & Cabeçalho 2 & Cabeçalho 3 \\
    \hline
    Texto longo que se ajusta automaticamente & Dado 2 & Dado 3 \\
    \hline
\end{tabularx}
\end{lstlisting}

  \begin{block}{Exemplo de tabela com \texttt{tabularx}}
    \centering
    \begin{tabularx}{\textwidth}{|X|c|r|} % Colunas X dividem o espaço igualmente
      \hline
      Cabeçalho 1 & Cabeçalho 2 & Cabeçalho 3 \\
      \hline
      Texto longo que se ajusta automaticamente & Dado 2 & Dado 3 \\
      \hline
    \end{tabularx}
  \end{block}

\end{frame}

\begin{frame}{Linhas, bordas e mesclagem}

  \begin{block}{Linhas horizontais e verticais}
    \begin{itemize}
      \item \texttt{\textbackslash hline}: Linha horizontal
      \item {cline\{i-j\}}: Linha horizontal parcial
      \item \texttt{|}: Bordas verticais (ex: \texttt{|c|c|c|})
    \end{itemize}
  \end{block}

  \begin{block}{Mesclando células}
    \begin{itemize}
      \item \texttt{\textbackslash multicolumn\{n\}\{alinhamento\}\{Título\}}: Mescla colunas
      \item \texttt{\textbackslash multirow\{n\}\{largura\}\{text\}}: Mescla linhas (requer o pacote \texttt{multirow}
      \end{itemize}
    \end{block}

  \end{frame}

  \begin{frame}[fragile]{Linhas, bordas e mesclagem}

\begin{lstlisting}[language=tex]
    \begin{center}
        \begin{tabular}{|l|c|r|}
            \hline
            \multicolumn{3}{|c|}{Título da Tabela} \\
            \hline
            Item & Quantidade & Preço \\
            \cline{1-2} % Linha parcial
            Livro & 2 & R\$ 100,00 \\
            \hline
        \end{tabular}
    \end{center}
\end{lstlisting}

    \begin{block}{Exemplo}
      \begin{center}
        \begin{tabular}{|l|c|r|}
          \hline
          \multicolumn{3}{|c|}{Título da Tabela} \\
          \hline
          Item & Quantidade & Preço \\
          \cline{1-2} % Linha parcial
          Livro & 2 & R\$ 100,00 \\
          \hline
        \end{tabular}
      \end{center}
    \end{block}

  \end{frame}

  % -----------
  % FIGURAS – para classes gerais (article, report, etc)
  % -----------

  \begin{frame}[fragile]{Inserindo figurar som o pacote \texttt{graphicx}}
    O pacote \texttt{graphicx} é \textbf{essencial} para inserir imagens em \LaTeX{}.

    Para carregar, coloque no pre-âmbulo:
    \begin{lstlisting}[language=tex]
      \usepackage{graphicx} % Sempre adicione isso!
    \end{lstlisting}

    Formatos suportados: \texttt{PNG}, \texttt{JPG}, \texttt{PDF}. (Evite \texttt{SVG} direto.)
  \end{frame}

  \begin{frame}[fragile]{Inserindo uma imagem}
    Use \texttt{\textbackslash includegraphics} para adicionar uma imagem:
    \begin{lstlisting}[language=tex]
      \includegraphics[width=0.5\textwidth]{caminho/para/imagem.png}
    \end{lstlisting}
  \end{frame}

  \begin{frame}[fragile]{Redimensionamento e rotação}
    \begin{itemize}
      \item Largura fixa: \texttt{width=8cm}
      \item Altura fixa: \texttt{height=4cm}
      \item Escala proporcional: \texttt{scale=0.7}
    \end{itemize}
    \begin{lstlisting}[language=tex, caption=Customizando tamanho e ângulo]
      \includegraphics[width=5cm, angle=30]{imagem.jpg}
    \end{lstlisting}
  \end{frame}

  \begin{frame}[fragile]{Ambiente figure: legenda e referência}
    O ambiente \texttt{figure} permite posicionamento, legenda e referência:
    \begin{lstlisting}[language=tex, caption=Imagem com legenda e referência]
      \begin{figure}[h!]
        \centering
        \includegraphics[width=0.4\textwidth]{caminho/para/imagem}
        \caption{Legenda opcional}
        \label{fig:exemplo}
      \end{figure}
    \end{lstlisting}
  \end{frame}

  % --------------------------
  % Modos matemáticos (básico)
  % --------------------------

  \begin{frame}{Modos matemáticos no \LaTeX{}}
    \begin{itemize}
      \item \textbf{Inline (no texto):} Use \texttt{\textbackslash(...\textbackslash)}| ou \texttt{\$...\$}.
      \item \textbf{Display (destacado):} Use \texttt{\textbackslash[...\textbackslash]} ou \texttt{\$\$...\$\$}.
    \end{itemize}

    \begin{block}{Exemplo inline}
      A equação $E = mc^2$ foi proposta por Einstein.
    \end{block}
    \begin{block}{Exemplo display}
      Equação proposta por Einstein: \[E = mc^2\]
    \end{block}

    \begin{block}{Equação referenciada}
      Lei Circuital de Ampère
      \begin{equation}
        \oint B \,\cdot\, dl \,=\, \mu\sum_{k=1}^{n} i_k \,=\, \mu I
      \end{equation}
    \end{block}

    \vspace{1em}
    Veja o PDF de símbolos matemáticos no drive do minicurso. No livro, este tema é explorado em muito mais detalhes!
  \end{frame}
