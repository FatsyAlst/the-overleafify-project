\section{Formatação de texto}

\begin{frame}[fragile]{Formatação básica}
  \begin{itemize}
    \item \textbf{Negrito}: \verb|\textbf{Texto em negrito}|

      \begin{itemize}
        \item O comando de atalho é geralmente: \textbf{CTRL + B}
      \end{itemize}

    \item \textit{Itálico}: \verb|\textit{Texto em itálico}|

      \begin{itemize}
        \item O comando de atalho é geralmente: \textbf{CTRL + I}
      \end{itemize}

    \item \underline{Sublinhado}: \verb|\underline{Texto sublinhado}|

      \begin{itemize}
        \item O comando de atalho é geralmente: \textbf{CTRL + U}
      \end{itemize}

  \end{itemize}
\end{frame}

\begin{frame}{Tamanhos de fonte}
  \begin{block}{Pré-definidos}
    Do menor ao maior: \texttt{\textbackslash tiny}, \texttt{\textbackslash scriptsize}, \texttt{\textbackslash footnotesize}, \texttt{\textbackslash small}, \texttt{\textbackslash normalsize}, \texttt{\textbackslash large}, \texttt{\textbackslash Large}, \texttt{\textbackslash LARGE}, \texttt{\textbackslash huge}, \texttt{\textbackslash Huge}
  \end{block}

  \begin{block}{Específico}
    \texttt{\{\textbackslash fontsize\{size\}\{skip\}\textbackslash selectfont Texto aqui\}}
  \end{block}

  \begin{block}{Exemplos}
    \begin{itemize}
      \item {\fontsize{10}{12}\selectfont Texto aqui}
      \item {\tiny tiny}, {\scriptsize scriptsize}, {\footnotesize footnotesize}, {\small small}, {\normalsize normalsize}, {\large large}, {\Large Large}, {\LARGE LARGE}, {\huge huge}, {\Huge Huge}
    \end{itemize}
  \end{block}

\end{frame}

\begin{frame}[fragile]{Cores no texto}
  Usamos o pacote \verb|xcolor|:
    \begin{lstlisting}[language=tex, caption=Uso do pacote \texttt{xcolor} para customização de cores no texto]
        \usepackage{xcolor}
        \textcolor{red}{Texto vermelho}
        \textcolor{blue}{Texto azul}
        \textcolor[HTML]{00FF00}{Verde em hexadecimal}
    \end{lstlisting}

  \textcolor{red}{Texto vermelho}
  \textcolor{blue}{Texto azul}
  \textcolor[HTML]{00FF00}{Verde em hexadecimal}
\end{frame}

\begin{frame}[fragile]{Listas}
    \begin{lstlisting}[language=tex]
        \begin{enumerate}
            \item Primeiro item
            \item Segundo item
        \end{enumerate}
    \end{lstlisting}

  \begin{enumerate}
    \item Primeiro item
    \item Segundo item
  \end{enumerate}

    \begin{lstlisting}[language=tex]
        \begin{itemize}
            \item Item com marcador padrão
            \item Outro item
        \end{itemize}
    \end{lstlisting}

  \begin{itemize}
    \item Item com marcador padrão
    \item Outro item
  \end{itemize}
\end{frame}

\begin{frame}{Quebras de linha, parágrafos e espaçamentos}

  \begin{block}{Exemplos}
    \begin{itemize}
      \item Quebra de linha: \texttt{\textbackslash\textbackslash} ou \texttt{\textbackslash newline}
      \item Novo parágrafo: Deixe uma linha vazia no código ou use \texttt{\textbackslash par}
      \item Espaçamento vertical: \texttt{\textbackslash vspace\{1cm\}}
      \item Espaçamento horizontal: \texttt{\textbackslash hspace\{2em\}}
      \item Pular pagina: \texttt{\textbackslash newpage}
    \end{itemize}
  \end{block}

\end{frame}
