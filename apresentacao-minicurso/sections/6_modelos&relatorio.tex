\section{Modelo e templates}

\begin{frame}{Principais fontes de templates no Overleaf}
  \begin{itemize}
    \item \textbf{Galeria oficial do Overleaf} \\
      \href{https://www.overleaf.com/gallery}{overleaf.com/gallery}
      \begin{itemize}
        \item Templates para TCCs, dissertações, relatórios, artigos, etc.
        \item Filtros por tipo de documento
      \end{itemize}

    \item \textbf{CTAN – Repositório oficial do LaTeX} \\
      \href{https://ctan.org}{ctan.org}
      \begin{itemize}
        \item Classes como \texttt{abnTeX2}, \texttt{UnBTeX}, etc.
        \item Templates para dissertações e TCCs com normas ABNT
        \item Compatível com Overleaf (pacotes via TeX Live)
      \end{itemize}
  \end{itemize}
\end{frame}

\begin{frame}{Mais fontes úteis de templates}
  \begin{itemize}
    \item \textbf{Universidades brasileiras}
      \begin{itemize}
        \item Muitas instituições disponibilizam modelos oficiais
        \item Exemplos: USP (Pacote USPSC), UnB (UnBTeX), IFs, UFs...
        \item Verifique o site da biblioteca ou departamento
      \end{itemize}

    \item \textbf{LaTeXTemplates.com} \\
      \href{https://www.latextemplates.com}{latextemplates.com}
      \begin{itemize}
        \item Templates atualizados
        \item Seções específicas: relatórios, teses, apresentações
        \item Pode ser usado no Overleaf sem problemas
      \end{itemize}
  \end{itemize}

  Também existem inúmeras outras fontes na internet, como comunidades nichadas no Reddit, Discord, outros fóruns, etc.

\end{frame}
