\section{Dúvidas frequentes e dicas finais}

\begin{frame}{Dicas e dúvidas}
  \begin{block}{Organize seu projeto em arquivos separados}
    \begin{itemize}
      \item Divida capítulos/partes em arquivos menores e depois insira-os no arquivo principal (\texttt{main.tex}):\\
        \texttt{\textbackslash include\{introducao.tex\}}\\
        \texttt{\textbackslash include\{capitulo1.tex\}}\\
        \texttt{\textbackslash include\{capitulo2.tex\}}
      \item Vantagens: compilação mais rápida, rastreio de mudanças (controle de versão) e colaboração sem conflitos.
      \item Organize seus arquivos em pastas.
    \end{itemize}
  \end{block}
\end{frame}

\begin{frame}{Dicas e dúvidas}
  \begin{block}{Ferramentas complementares úteis}
    \begin{itemize}
      \item \textbf{Detexify} – desenhe um símbolo à mão e receba o comando LaTeX correspondente.
      \item \textbf{ChatGPT (normal ou modelo específico)} – peça explicações de erros de compilação, gere exemplos mínimos ou descubra pacotes adequados para uma tarefa específica.
    \end{itemize}
  \end{block}

  \vspace{0.3cm}
  \textbf{OBS:} O compilador online Crixet já possui funções semelhantes embutidas nativamente!

\end{frame}

\begin{frame}{Dicas e dúvidas}
  \begin{block}{Boas práticas}
    \begin{itemize}
      \item \textbf{Compile com frequência}: encontre erros logo após escrevê‑los, quando a causa ainda está clara.
      \item \textbf{Comente seu código} (\texttt{\%}): explique trechos longos, marque \emph{TODOs} e facilite revisões futuras.
      \item \textbf{Bibliotecas de Padrões}: use bibliotecas como o abntex2 e IEEEtran, para ajustar os padrões do seu documento.
    \end{itemize}
  \end{block}
\end{frame}

\begin{frame}{Dicas e dúvidas}
  \begin{block}{Além dos mitos e verdades}
    \begin{itemize}
      \item A curva de aprendizado existe, mas se paga rapidamente quando projetos crescem e você passa a economizar tempo com o LaTeX.
      \item Ninguém domina todos os pacotes – a produtividade surge de conhecer \emph{bem} o conjunto que você usa e de saber onde encontrar ajuda (documentação, comunidade, IA).
      \item Separar conteúdo de forma (princípio WYSIWYM) continua sendo a resposta moderna às crises de formatação que assombram editores WYSIWYG.
    \end{itemize}
  \end{block}

  \vfill

  \begin{quote}
    \emph{''Aprender \LaTeX{} não é um evento pontual, mas um processo contínuo. A boa notícia é que você começa a colher os frutos muito antes de “chegar ao final — porque, na verdade, não há final.''}
  \end{quote}
\end{frame}

%-----------------------------------------------------------------
