% Tema
\usetheme{Madrid}

% Pacotes adicionais
\usepackage[utf8]{inputenc}
\usepackage{graphicx}
\usepackage{xcolor}
\usepackage[brazilian]{babel}
\usepackage{subcaption}
\usepackage{verbatim}
\usepackage{tabularx} % Tabelas com largura ajustável

\usepackage{xspace}  % Carrega o pacote xspace, que insere automaticamente um espaço após macros
\newcommand{\LaTeXcmd}{\LaTeX\xspace} % Define \LaTeXcmd como um atalho para \LaTeX, e usa \xspace para adicionar um espaço quando o próximo caractere for alfanumérico, evitando precisar escrever {} ou \, toda vez.
\newcommand{\TeXcmd}{\TeX\xspace} % Define \TeXcmd análogo a \LaTeXcmd, mas para o logo \TeX.

% AMBIENTE PARA CÓDIGO-FONTE
%--------------------------------------------------------
\usepackage{listings} % Permite incluir e formatar código-fonte
    \lstset{
        language=tex,               % Linguagem usada no código (aqui: LaTeX)
        basicstyle=\ttfamily\small, % Fonte monoespaçada pequena
        commentstyle=\color{gray},  % Cor dos comentários
        frame=single,               % Moldura em volta do código
        breaklines=true,            % Quebra linhas automaticamente
        captionpos=b                % Posição da legenda (b = abaixo)
    }

\renewcommand{\lstlistingname}{Código} % Muda o nome padrão de "Listing" para "Código"

    % Suporte para acentuação correta dentro do ambiente de código
    % Resolve um tipo de erro bem especifico em blocos de codigo
    \lstset{
        literate={á}{{\'a}}1 {é}{{\'e}}1 {í}{{\'i}}1 {ó}{{\'o}}1 {ú}{{\'u}}1
                 {ã}{{\~a}}1 {õ}{{\~o}}1 {â}{{\^a}}1 {ê}{{\^e}}1 {ô}{{\^o}}1
                 {À}{{\`A}}1 {à}{{\`a}}1 {ü}{{\"u}}1 {ç}{{\c{c}}}1 {Ç}{{\c{C}}}1
                 {ö}{{\"o}}1
    }